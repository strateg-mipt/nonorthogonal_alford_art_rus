\section{Описание модельных задач}

\begin{table}
\footnotesize
\centering
\caption{Параметры упругих анизотропных материалов}
\renewcommand{\arraystretch}{1.5}
\begin{tabularx}{\textwidth}{|C|c|c|c|c|c|c|c|}
\hline \multirow{2}{*}{Название}  & Плотность & \multicolumn{6}{c|}{Упругие модули, ГПа} \\ 
\cline{3-8}  & кг/м$^3$ & $C_{11}$ & $C_{12}$ & $C_{13}$ & $C_{33}$ & $C_{44}$ & $C_{66}$ \\ \hline
\hline Austin Chalk & 2200 & 22 & 15.8 & 12 & 14 & 2.4 & 3.1 \\ 
\hline Cotton Valey Shale & 2640 & 74.73 & 14.75 & 25.29 & 58.84 & 22.05 & 29.99 \\ 
\hline Bakken Shale & 2230 & 40.9 & 10.3 & 8.5 & 26.9 & 10.5 & 15.3 \\ 
\hline 
\end{tabularx} 
\renewcommand{\arraystretch}{1.0}
\end{table}

\begin{table}
\footnotesize
\centering
\caption{Параметры модельных задач}
\renewcommand{\arraystretch}{1.5}
\begin{tabularx}{\textwidth}{|C|l|c|l|c|c|}
\hline  Обозначение & Форма скважины & Геометрия & Материал породы & TI угол $\zeta$ & TI угол $\phi$ \\ \hline
\hline \textbf{}\modelnum{\label{mnum: 1}}& Цилиндрическая & $20 \times 20$ см & Austin Chalk & 60 & 45 \\ 
\hline \textbf{}\modelnum{\label{mnum: 2}}& Эллиптическая & $15 \times 10$ см & Изотроп. медленная & - & - \\ 
\hline \textbf{}\modelnum{\label{mnum: 3}}& Эллиптическая & $20 \times 10$ см & Austin Chalk & 60 & 45 \\ 
\hline \textbf{}\modelnum{\label{mnum: 4}}& Эллиптическая & $15 \times 10$ см & Austin Chalk & 60 & 45 \\ 
\hline \textbf{}\modelnum{\label{mnum: 5}}& Эллиптическая & $15 \times 10$ см & Cotton Valey Shale & 60 & 45 \\ 
\hline \textbf{}\modelnum{\label{mnum: 6}}& Эллиптическая & $20 \times 10$ см & Cotton Valey Shale & 60 & 45 \\ 
\hline \textbf{}\modelnum{\label{mnum: 7}}& Цилиндрическая & $15 \times 15$ см & Cotton Valey Shale & 60 & 45 \\ 
\hline \textbf{}\modelnum{\label{mnum: 8}}& Эллиптическая & $15 \times 10$ см & Изотропная & - & - \\ 
\hline \textbf{}\modelnum{\label{mnum: 9}}& Эллиптическая & $15 \times 10$ см & Cotton Valey Shale & 90 & 45 \\
\hline \textbf{}\modelnum{\label{mnum: 10}}& Эллиптическая & $15 \times 10$ см & Bakken Shale & 90 & 45 \\
\hline \textbf{}\modelnum{\label{mnum: 11}}& Эллиптическая & $10 \times 8$ дьюм & Bakken Shale & 90 & 45 \\
\hline \textbf{}\modelnum{\label{mnum: 12}}& Эллиптическая & $10 \times 8$ дьюм & Bakken Shale & 90 & 90 \\
\hline 
\end{tabularx}
\renewcommand{\arraystretch}{1.0}
\end{table}