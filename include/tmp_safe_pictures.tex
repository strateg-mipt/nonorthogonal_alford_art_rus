\begin{figure}[h]
\centering
\begin{minipage}{0.24\linewidth}
	\psfragfig[width=0.24\linewidth,crop=pdfcrop]{./images/SAFE/CS_15x10/cs15x10_sym_f1_5kHz+ang}		
\end{minipage}
\begin{minipage}{0.24\linewidth}
	\psfragfig[width=0.24\linewidth,crop=pdfcrop]{./images/SAFE/CS_15x10/cs15x10_sym_f2_39kHz+ang}		
\end{minipage}
\begin{minipage}{0.24\linewidth}
	\psfragfig[width=0.24\linewidth,crop=pdfcrop]{./images/SAFE/CS_15x10/cs15x10_sym_f3_28kHz+ang}		
\end{minipage}
\begin{minipage}{0.24\linewidth}
	\psfragfig[width=0.24\linewidth,crop=pdfcrop]{./images/SAFE/CS_15x10/cs15x10_sym_f5_52kHz+ang}		
\end{minipage}
\hfill
\begin{minipage}{0.24\linewidth}
	\psfragfig[width=0.24\linewidth,crop=pdfcrop]{./images/SAFE/CS_15x10/cs15x10_asym_f1_5kHz+ang}		
\end{minipage}
\begin{minipage}{0.24\linewidth}
	\psfragfig[width=0.24\linewidth,crop=pdfcrop]{./images/SAFE/CS_15x10/cs15x10_asym_f2_39kHz+ang}		
\end{minipage}
\begin{minipage}{0.24\linewidth}
	\psfragfig[width=0.24\linewidth,crop=pdfcrop]{./images/SAFE/CS_15x10/cs15x10_asym_f3_28kHz+ang}		
\end{minipage}
\begin{minipage}{0.24\linewidth}
	\psfragfig[width=0.24\linewidth,crop=pdfcrop]{./images/SAFE/CS_15x10/cs15x10_asym_f5_52kHz+ang}		
\end{minipage}
\hfill
\vspace{\baselineskip}
\begin{minipage}{0.24\linewidth}
	\psfragfig[width=0.24\linewidth,crop=pdfcrop]{./images/SAFE/CS_15x10/p_cs15x10_sym_f1_5kHz+ang}		
\end{minipage}
\begin{minipage}{0.24\linewidth}
	\psfragfig[width=0.24\linewidth,crop=pdfcrop]{./images/SAFE/CS_15x10/p_cs15x10_sym_f2_39kHz+ang}		
\end{minipage}
\begin{minipage}{0.24\linewidth}
	\psfragfig[width=0.24\linewidth,crop=pdfcrop]{./images/SAFE/CS_15x10/p_cs15x10_sym_f3_28kHz+ang}		
\end{minipage}
\begin{minipage}{0.24\linewidth}
	\psfragfig[width=0.24\linewidth,crop=pdfcrop]{./images/SAFE/CS_15x10/p_cs15x10_sym_f5_52kHz+ang}		
\end{minipage}
\hfill
\begin{minipage}{0.24\linewidth}
	\psfragfig[width=0.24\linewidth,crop=pdfcrop]{./images/SAFE/CS_15x10/p_cs15x10_asym_f1_5kHz+ang}		
\end{minipage}
\begin{minipage}{0.24\linewidth}
	\psfragfig[width=0.24\linewidth,crop=pdfcrop]{./images/SAFE/CS_15x10/p_cs15x10_asym_f2_39kHz+ang}		
\end{minipage}
\begin{minipage}{0.24\linewidth}
	\psfragfig[width=0.24\linewidth,crop=pdfcrop]{./images/SAFE/CS_15x10/p_cs15x10_asym_f3_28kHz+ang}		
\end{minipage}
\begin{minipage}{0.24\linewidth}
	\psfragfig[width=0.24\linewidth,crop=pdfcrop]{./images/SAFE/CS_15x10/p_cs15x10_asym_f5_52kHz+ang}		
\end{minipage}
\hfill

\caption{Графическое представление собственных векторов дипольных мод внутри (поле давления) и снаружи (компонента смещения z) скважины. Модель \ref{mnum: 5}. Здесь сплошные линии указывают направления, полученные неортогональным Alford rotation, примененным к исходным данным; прерывистая линия - к данным с оконной фильтрацией, прерывистая с точкой - к данным с низкочастотной фильтрацией.}
\end{figure}

\begin{figure}[h]
\centering
	\psfragfig[width=1\linewidth,crop=pdfcrop]{./images/SAFE/CS_15x10/modes_U+angles}		
	\caption{Представление значения $U = \sqrt{U_x^2+U_y^2+U_z^2}$ для собственных векторов в породе в зависимости от частоты. Модель \ref{mnum: 5}, быстрая порода. }
\end{figure}

\begin{figure}[h]
\centering
\begin{minipage}{0.24\linewidth}
	\psfragfig[width=0.24\linewidth,crop=pdfcrop]{./images/SAFE/AC_15x10/ac15x10_sym_f1_5kHz+ang}		
\end{minipage}
\begin{minipage}{0.24\linewidth}
	\psfragfig[width=0.24\linewidth,crop=pdfcrop]{./images/SAFE/AC_15x10/ac15x10_sym_f3_28kHz+ang}		
\end{minipage}
\begin{minipage}{0.24\linewidth}
	\psfragfig[width=0.24\linewidth,crop=pdfcrop]{./images/SAFE/AC_15x10/ac15x10_sym_f5_52kHz+ang}		
\end{minipage}
\begin{minipage}{0.24\linewidth}
	\psfragfig[width=0.24\linewidth,crop=pdfcrop]{./images/SAFE/AC_15x10/ac15x10_sym_f10_0kHz+ang}		
\end{minipage}
\hfill
\begin{minipage}{0.24\linewidth}
	\psfragfig[width=0.24\linewidth,crop=pdfcrop]{./images/SAFE/AC_15x10/ac15x10_asym_f1_5kHz+ang}		
\end{minipage}
\begin{minipage}{0.24\linewidth}
	\psfragfig[width=0.24\linewidth,crop=pdfcrop]{./images/SAFE/AC_15x10/ac15x10_asym_f3_28kHz+ang}		
\end{minipage}
\begin{minipage}{0.24\linewidth}
	\psfragfig[width=0.24\linewidth,crop=pdfcrop]{./images/SAFE/AC_15x10/ac15x10_asym_f5_52kHz+ang}		
\end{minipage}
\begin{minipage}{0.24\linewidth}
	\psfragfig[width=0.24\linewidth,crop=pdfcrop]{./images/SAFE/AC_15x10/ac15x10_asym_f10_0kHz+ang}		
\end{minipage}
\hfill
\vspace{\baselineskip}
\begin{minipage}{0.24\linewidth}
	\psfragfig[width=0.24\linewidth,crop=pdfcrop]{./images/SAFE/AC_15x10/p_ac15x10_sym_f1_5kHz+ang}		
\end{minipage}
\begin{minipage}{0.24\linewidth}
	\psfragfig[width=0.24\linewidth,crop=pdfcrop]{./images/SAFE/AC_15x10/p_ac15x10_sym_f3_28kHz+ang}		
\end{minipage}
\begin{minipage}{0.24\linewidth}
	\psfragfig[width=0.24\linewidth,crop=pdfcrop]{./images/SAFE/AC_15x10/p_ac15x10_sym_f5_52kHz+ang}		
\end{minipage}
\begin{minipage}{0.24\linewidth}
	\psfragfig[width=0.24\linewidth,crop=pdfcrop]{./images/SAFE/AC_15x10/p_ac15x10_sym_f10_0kHz+ang}		
\end{minipage}
\hfill
\begin{minipage}{0.24\linewidth}
	\psfragfig[width=0.24\linewidth,crop=pdfcrop]{./images/SAFE/AC_15x10/p_ac15x10_asym_f1_5kHz+ang}		
\end{minipage}
\begin{minipage}{0.24\linewidth}
	\psfragfig[width=0.24\linewidth,crop=pdfcrop]{./images/SAFE/AC_15x10/p_ac15x10_asym_f3_28kHz+ang}		
\end{minipage}
\begin{minipage}{0.24\linewidth}
	\psfragfig[width=0.24\linewidth,crop=pdfcrop]{./images/SAFE/AC_15x10/p_ac15x10_asym_f5_52kHz+ang}		
\end{minipage}
\begin{minipage}{0.24\linewidth}
	\psfragfig[width=0.24\linewidth,crop=pdfcrop]{./images/SAFE/AC_15x10/p_ac15x10_asym_f10_0kHz+ang}		
\end{minipage}
\hfill

\caption{Графическое представление собственных векторов дипольных мод внутри (поле давления) и снаружи (компонента смещения z) скважины. Модель \ref{mnum: 3}. Здесь сплошные линии указывают направления, полученные неортогональным Alford rotation, примененным к исходным данным; прерывистая линия - к данным с низкочастотной фильтрацией, прерывистая с точкой - к данным с высокочастотной фильтрацией.}
\end{figure}

\begin{figure}[h]
\centering
	\psfragfig[width=1\linewidth,crop=pdfcrop]{./images/SAFE/AC_15x10/ac15x10_modes_U+angles}
	\caption{Представление значения $U = \sqrt{U_x^2+U_y^2+U_z^2}$ для собственных векторов в породе в зависимости от частоты. Модель \ref{mnum: 3}, медленная порода. }		
\end{figure}

\textbf{Зависимость энергии диагональных компонент от значений $\theta$ и $\eta$}\\
			\psfragfig[width=0.40\linewidth,crop=pdfcrop]{./images/nonorth_alford/solution_min_gs_rot4c}
			\label{fig:rot4_gs_solution}

\begin{minipage}[h]{0.47\linewidth}
\begin{center}
\textbf{Результат работы TKO}
			\psfragfig[width=0.40\linewidth,crop=pdfcrop]{./images/nonorth_alford/el20x10_TTI60_TKO_compare}\\
	  		\label{fig:rot4_tko_comp}
\end{center}	  		
\end{minipage}

\begin{figure}[h]
\centering
\begin{minipage}{0.47\linewidth}
	\psfragfig[width=0.47\linewidth,crop=pdfcrop]{./images/SAFE/CS_15x10/cs15x10_sym_f1_5kHz_U+ang}		
\end{minipage}
\hfill
\begin{minipage}{0.47\linewidth}
	\psfragfig[width=0.47\linewidth,crop=pdfcrop]{./images/SAFE/CS_15x10/cs15x10_sym_f7_76kHz_U+ang}		
\end{minipage}
\vspace{\baselineskip}
\begin{minipage}{0.47\linewidth}
	\psfragfig[width=0.47\linewidth,crop=pdfcrop]{./images/SAFE/CS_15x10/cs15x10_asym_f1_5kHz_U+ang}		
\end{minipage}
\hfill
\begin{minipage}{0.47\linewidth}
	\psfragfig[width=0.47\linewidth,crop=pdfcrop]{./images/SAFE/CS_15x10/cs15x10_asym_f7_76kHz_U+ang}		
\end{minipage}
\caption{Представление значения $|U|$ для собственных векторов в породе в зависимости от частоты. Модель \ref{mnum: 5}, быстрая порода. Здесь сплошные линии указывают направления, полученные неортогональным Alford rotation, примененным к исходным данным; прерывистая линия - к данным с оконной фильтрацией, прерывистая с точкой - к данным с низкочастотной фильтрацией. }	
\end{figure}

\begin{figure}[h]
\begin{minipage}{1\linewidth}
	\psfragfig[width=1\linewidth,crop=pdfcrop]{./images/SAFE/CS_15x10/disp_close_spec}		
\end{minipage}
\vspace{\baselineskip}
\begin{minipage}{1\linewidth}
	\psfragfig[width=1\linewidth,crop=pdfcrop]{./images/SAFE/CS_15x10/modes_close_spec}		
\end{minipage}
\end{figure}