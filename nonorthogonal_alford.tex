% !TeX spellcheck = ru_RU
\documentclass[a4paper,11pt]{article}
\usepackage[process=auto]{pstool}
\usepackage[english,russian]{babel}
\usepackage[T2A]{fontenc}
\usepackage[utf8]{inputenc}
\usepackage{amssymb,amsmath}
\usepackage{gensymb,textcomp,latexsym}
\usepackage{graphicx}
\usepackage{tabularx}
\usepackage[pdftex, left=1in, right=1in, top=1in, bottom=2cm]{geometry}
\usepackage{parcolumns}
\usepackage{multirow}
\usepackage{tikz}

%\usepackage[usenames,dvipsnames]{xcolor}
%\usepackage[font=small,labelfont=bf]{caption}
\usepackage[center]{subfigure}
\renewcommand{\thesubfigure}{(\asbuk{subfigure})~}
\newcommand{\figref}[1]{Рис.~\ref{#1}}
\usepackage[pdfauthor={Shchelik},pdftitle={Nonorthogonal Alford test},pdfstartview=XYZ,bookmarks=true,colorlinks=true,linkcolor=blue,urlcolor=blue,citecolor=blue,
bookmarks=true,linktocpage=true,hyperindex=true]{hyperref}
%\usepackage[hyperpageref]{backref}

%\usepackage[section]{placeins}
%\usepackage{graphicx}
%\usepackage{epsfig}
%\usepackage{epstopdf}
%\usepackage{subfigure}
\usepackage{float}
%\usepackage{booktabs}
%\graphicspath{ {./image_test/} }

\newcounter{modelnum}
\newcommand{\modelnum}[1]{\refstepcounter{modelnum}Модель \themodelnum #1}

%Filters counters
\newcounter{wfiltnum}
\newcommand{\wfiltnum}[1]{\refstepcounter{wfiltnum}ОФ-\thewfiltnum #1}
\newcounter{lffiltnum}
\newcommand{\lffiltnum}[1]{\refstepcounter{lffiltnum}НЧФ-\thelffiltnum #1}
\newcounter{hffiltnum}
\newcommand{\hffiltnum}[1]{\refstepcounter{hffiltnum}ВЧФ-\thehffiltnum #1}
%tabularx options
\newcolumntype{C}{>{\centering}X}


\begin{document}
\part*{Анализ поляризации дипольных волн в нецилиндрических скважинах в анизотропной породе}
\today

%К вопросу о поляризации дипольных мод в нецилиндрических скважинах в анизотропной породе.
%
%- Введение
%    Обсуждение методики дипольного каротажа, задача определения направлений основных осей трансверсально-анизотропной породы. Обзор темы существования главных направлений в нецилиндрических скважинах. Феномен неортогональности, алгоритмы разделения мод. 
%
%- Дипольные моды в анизотропной среде. Дисперсионный подход, стандартный алгоритм Alford rotation. Феномен неортогональности, алгоритмы разделения мод. 
%- Неортогональность главных направлений и альтернативный метод обработки NOAR. Зависимость от частоты. Фильтрация сигнала.
%- SAFE как метод изучения отдельных базисных решений. Описание и применение
%
%- Сравнение результатов SAFE и NOAR.
%
%- Заключение.

\section{Введение}
Уже несколько десятилетий в акустическом каротаже широко и достаточно успешно практикуются методики кросс-дипольных измерений. Современные решения в технике и обработке полученных данных позволяют количественно производить оценку азимутальной и аксиальной (по отношению к стволу скважины) анизотропии для широкой группы горных пород. Существенный прогресс в определении параметров породы в рамках модели трансверсально-изотропного (ТИ) тела по скоростям распространения поперечных волн называется рядом специалистов главным достижением методов акустического каротажа последних лет. Кросс-дипольные измерения также могут быть использованы для определения ориентации крупных трещин и обнаружения анизотропии, индуцированной подземными горизонтальными напряжениями \cite{Patterson2001}.

Как известно, принцип работы данного метода основан на существовании в ТИ породе двух ортогональных выделенных направлений, что в общем случае приводит к поляризации распространяющихся по стволу скважины поперечных волн. При наличии измерений от двух направленных ортогонально-ориентированных источников в скважине возможно определить направления главных осей ТИ модели и скорости распространения поляризованных поперечных волн. В основе классического метода определения лежит допущение о симметричности матрицы измерений (составленной из данных четырёх измерений с различной ориентацией источников и приемников), которая может быть приведена к диагональному виду ортогональным поворотом на некоторый угол \cite{Alford1986}. В ряде практических случаев ортогональность направлений поляризации поперечных волн отсутствует, но в однородной породе симметрия матрицы измерений сохраняется и существуют возможности её диагонализации другими способами \cite{Dellinger1998}. Помимо неортогональности на результат работы классического алгоритма может влиять наличие сильных горизонтальных напряжений, приводящих к возникновению индуцированной анизотропии и эффекту пересечения дисперсионных кривых быстрой и медленной волн. Результаты работы алгоритма становятся при этом зависимыми от ширины выбранного временного окна, что свидетельствует о зависимости направления поляризации волн от частоты \cite{Nolte1997}. Другим фактором, влияющим на поляризацию и разделение волн в скважине, является неидеально цилиндрическая форма поперечного сечения ствола \cite{Seroices2010}. В сочетании с анизотропными свойствами горной породы отклонения от цилиндрической формы могут приводить к проблемам определения направлений главных осей стандартными методами [?].

В данной работе рассматривается вопрос определения главных направлений трансверсально-изотропной породы по измерениям в скважинах эллиптического сечения, являющихся модельным приближением более общих несимметричных форм скважин. В качестве исходных данных используются результаты численного трёхмерного моделирования с помощью метода спектральных элементов \cite{Komatitsch2000}. Для анализа структуры волнового поля в скважинах на определенной частоте применяется полуаналитический метод конечных элементов (SAFE) \cite{Bartoli2006}. Результаты расчётов сопоставляются с данными, получаемыми классическим алгоритмом Alford rotation \cite{Alford1986} и его альтернативной неортогональной модификации \cite{Dellinger1998} с применением оконной и частотной фильтрации и без неё.

\section{Дипольный каротаж скважин и классический подход к обработке}
Классическая схема работы современных приборов дипольного каротажа включает в себя записи от двух источников, ориентированных в некоторых направлениях X и Y, ортогональных друг другу. Запись производится на два массива приёмников, расположенных на разных уровнях с фиксированным шагом смещения вдоль прибора. Таким образом, на каждом уровне устройство формирует четырехкомпонентную матрицу измерений, состоящую из двух замеров по направлениям излучения источников (XX и YY) и перекрёстных замеров (XY и YX), где первая буква обозначает ориентацию источника, а вторая - приемников. Обозначим матрицу $2 \times 2$, состоящую из этих компонент, как $\mathbf{R}$ 

$$
	\mathbf{R} = \left\|
	\begin{array}{cc}
	\text{XX} & \text{YX} \\
	\text{XY} & \text{YY} \\
	\end{array}
	\right\| 
$$

Из решения уравнения Кельвина-Кристоффеля для породы с ТИ типом симметрии известно, что в направлении отличном от направления оси симметрии породы существует три решения для плоских волн (квазипродольная qP, квазипоперечная qSV и чисто поперечная SH волны), имеющих ортогональную поляризацию по отношению друг к другу. Дипольным источником внутри скважины в такой породе будут преимущественно возбуждаться две поперечные изгибные моды, обладающие сильной дисперсией, а в низкочастотном пределе имеющие скорость равную скорости qSV и SH волн в породе. Дисперсионные кривые этих мод не имеют пересечений и обе волны (обозначаемые как быстрая и медленная) распространяются независимо вдоль ствола скважины. В рамках классического подхода полагают, что для описания таких волновых процессов в скважине допустимо использование приближения, справедливого для распространения плоских волн в анизотропной недисперсионной среде, математически представленного в форме:

$$
	\mathbf{R} = \mathbf{P}_{M \rightarrow R} \ \mathbf{D} \ \mathbf{P}_{S \rightarrow M} \ \mathbf{S},
$$
где матрица $\mathbf{S}$ характеризует излучение источника, $\mathbf{P}_{S \rightarrow M}$ - матрица, проецирующая вектор источника на главные направления распространения нормальных дипольных мод, $\mathbf{P}_{M \rightarrow R}$ - матрица, проецирующая сигнал чистых мод на направления приемников, $\mathbf{D}$ - матрица, определяющая распространение чистых мод вдоль скважины. В предположении независимости распространения мод, матрица $\mathbf{D}$ будет диагональной. 

В системе координат, связанной с направлениями излучения дипольных источников, матрица $\mathbf{S}$ является единичной. Также если системы координат источников и приёмников совпадают, то $\mathbf{P}_{M \rightarrow R}=({\mathbf{P}_{S \rightarrow M}})^{T} = \mathbf{P}$. Для матрицы $\mathbf{D}$ в этом случае справедливо

\begin{gather}
	\mathbf{R} = \mathbf{P} \ \mathbf{D} \ \mathbf{P}^T, \label{eq:alford_symmetric} \\ 
	\mathbf{D} = \mathbf{P}^{-1} \ \mathbf{R} \ (\mathbf{P}^{-1})^{T}.
\end{gather}

Если направления поляризации дипольных волн ортогональны, то существует естественная система координат, связанная с главными направлениями и матрица преобразования $\mathbf{P}$ является поворотом на некоторый угол $\theta$. 

\begin{equation*}
	\mathbf{P} = \left\|
	\begin{array}{cc}
	\cos \theta &-\sin \theta \\ 
	\sin \theta & \cos \theta
	\end{array} 
	\right\| 
\end{equation*}

Алгоритм поиска этого угла, основанный на классической модели был представлен в работе \cite{Alford1986} и получил название Alford rotation. На практике полевые данные, однако, не обладают абсолютной симметрией и искомый угол находят из условия минимизации энергии недиагональных компонент по всему интервалу времени измерений. Следует отметить, что данный подход корректно работает только при сохранении ориентации главных направлений вдоль всего пути распространения волн, то есть предполагает однородность анизотропного материала на участке акустических измерений.

\section{Нарушение ортогональности}

При выводе основных положений классического Alford rotation был использован ряд предпосылок относительно свойств породы и модели распространения волн, накладывающих ограничения на использование метода. В частности, одним из недостатков такого подхода ряд исследователей называет требование ортогональной поляризации изгибных мод. Между тем в ряде случаев, например при распространении волн в анизотропной породе с орторомбическим типом симметрии \cite{Dellinger2001} или в случае анизотропии вызванной наличием трещин \cite{Nolte1996}, поляризация волн может быть существенно неортогональной. 

Один из возможных вариантов обобщения Alford rotation на неортогональный случай был рассмотрен в работе \cite{Dellinger1998} и заключается в введении дополнительного угла $\eta$, характеризующего ориентацию главных направлений. Утверждается, что если для системы справедливо представление \eqref{eq:alford_symmetric}, то матрицы преобразования будут иметь вид:
\begin{align*}
\mathbf{P} &= \left\|
\begin{array}{cc}
\cos \theta & -\sin (\theta+\eta) \\ 
\sin \theta & \cos (\theta+\eta)
\end{array} 
\right\| \\
\mathbf{P}^{-1} &= \frac{1}{\cos \eta} \left\|
\begin{array}{cc}
\cos (\theta+\eta) & \sin (\theta+\eta) \\ 
-\sin \theta & \cos (\theta)
\end{array} 
\right\|
\end{align*}
где, за $\theta$ принимается угол отсчитываемый против часовой стрелки между осью $X$ и направлением поляризации первой моды, а за $\theta + \eta$ - угол между направлением поляризации второй моды и осью $Y$. Поиск значений углов производится путем минимизация энергии недиагональных компонент матрицы $\mathbf{D}$ по двум параметрам. При $\eta=0$ метод сводится к классическому Alford rotation. 

Как известно, нарушение цилиндрической формы ствола скважины в изотропной породе также приводит к разделению изгибных мод и их поляризации вдоль направлений деформации \cite{Seroices2010}. Хотя влияние нециллиндричности практически сходит на нет для низкочастотных гармоник распространяющихся волн, на средних и высоких частотах этот фактор можно рассматривать как дополнительную причину анизотропии в случае, если главные направления анизотропной породы не совпадают с направлениями деформации ствола. Из-за частотной зависимости влияния геометрии и сильной дисперсии изгибных волн, эффективное направление поляризации может меняться по мере удаления приемников от источника. Это приводит к нарушению предположения классического метода (а также его неортогонального обобщения) о разделении изгибных мод, и как следствие симметрии матрицы измерений при любых ортогональных преобразованиях. 

Применение методов оконной и частотной фильтрации в ряде случаев может уменьшить ошибку работы классических алгоритмов. Однако универсального подхода, способного скорректировать влияние геометрии ствола на точность определения главных направлений анизотропной породы, в настоящее время не предложено. В следующем разделе производится оценка точности ортогонального и неортогонального Alford rotation на основе синтетических данных трехмерного моделирования распространения волн в быстрой ТИ породы в скважинах эллиптического сечения.  

\section{Вычислительные методы}

В качестве исходных данных каротажных измерений используются результаты прямого моделирования распространения волн методом спектральных элементов (SEM). Ранее данный метод успешно применялся для расчета задач геофизики \cite{Komatitsch1999} и моделирования акустического каротажа \cite{Charara2011}. Численный алгоритм производит решение уравнений твердого линейно-упругого анизотропного тела в твердых областях и акустических уравнений для невязкой жидкости внутри скважины с соответствующими условиями на границе раздела фаз. Подробное описание и формулировка метода приводится в работе \cite{Komatitsch1999}. Используемая в работе реализация производит основные расчеты на графических процессорах (GPU), что дает значительные преимущества в скорости расчета по сравнению с многими другими пакетами 3D-моделирования.

Основная обработка и анализ данных, в том числе и Alford rotation, проводились с помощью вычислительного комплекса MATLAB. Для построения дисперсионных кривых нормальных мод использовался алгоритм, основанный на модифицированном методе Прони \cite{Ekstrom1995}. При обработке данных измерений в некоторых случаях применялись оконные, низкочастотные и высокочастотные фильтры сигнала, реализованные стандартными возможностями вычислительного пакета.   

Для анализа решения в частотной области был выбран более простой и быстрый полуаналитический метод конечных элементов (SAFE) \cite{Bartoli2006}. Формулировка метода основана на возможности Фурье разложения искомой функции вдоль направления оси скважины, что позволяет свести задачу к двухмерной постановке. Ниже приведем краткое описание основные положений метода. Предполагая гармоническую зависимость вида $e^{-i\omega t}$ для смещений $\mathbf{u}$, деформаций $\boldsymbol{\varepsilon}$ и напряжений $\boldsymbol{\sigma}$, уравнения движения твёрдого тела в вариационной форме могут быть представлены в виде:

\begin{equation}
\int_{V}^{(s)}\delta \boldsymbol{\varepsilon}^* \boldsymbol{\sigma} dV - \omega^2 \int_{V}^{(s)} \rho_s \delta \mathbf{u}^*\mathbf{u}dV = \int_{V}^{(s)}\delta \mathbf{u}^* \mathbf{f} dV + \int_{\partial V}^{(s)}\delta \mathbf{u}^* \mathbf{t} d\Gamma, \label{var_eq_solid}
\end{equation}
здесь $\mathbf{f}$, $\mathbf{t}$ -- векторы объёмных и поверхностных сил, $\rho_s$ -- плотность, тензор напряжений связан с тензором деформаций для упругого тела через закон Гука:
$$
\boldsymbol{\sigma} = \mathbf{C}\boldsymbol{\varepsilon}.
$$

При описании движения невязкой жидкости будем пользоваться формулировкой уравнений в терминах потенциала скорости $\phi$: $\dot{\mathbf{u}}_f = - \nabla \phi$. Тогда давление в жидкости определяется выражением $p = \rho_f \dot{\phi}$. Уравнения движения для жидкой среды в вариационной форме 

\begin{equation}
\int_{V}^{(f)} \delta (\nabla\phi)^* \rho_f  \nabla \phi dV - \omega^2 \int_{V}^{(f)}  c^{-2} \rho_f \delta \phi^*  \phi dV = \frac{1}{i\omega}\int_{\partial V}^{(f)} \rho_f \delta(\nabla \phi)^* \mathbf{t} d\Gamma + \frac{1}{i\omega} \int_{V}^{(f)} \delta(\nabla \phi)^* \mathbf{f} dV, \label{var_eq_fluid}
\end{equation}
где $c = \sqrt{\lambda/\rho_f}$ -- скорость звука в жидкости.

Свяжем вертикальную ось скважины с направлением оси Z системы координат. Тогда значения искомых величин в плоскости поперечного сечения скважины можно аппроксимировать системой базисных функций на сетке конечных элементов \cite{Zienkiewicz2000}:
\begin{equation}
\begin{split}
\mathbf{u}(x,y,k,\omega) & = \mathbf{N}_u(x,y) \mathbf{U}^{(j)}(k,\omega) e^{i(kz-\omega t)} \\
\phi(x,y,k,\omega) & = \mathbf{N}_{\phi}(x,y) \mathbf{\Phi}^{(j)}(k,\omega) e^{i(kz-\omega t)}
\end{split} \label{unknown_var}
\end{equation}

С учётом условий на границе раздела жидкости и твёрдого тела при подстановке неизвестных \eqref{unknown_var} в уравнения \eqref{var_eq_solid} и \eqref{var_eq_fluid} задача может быть записана в форме \cite{Bartoli2006,Treyssede2013}:
\begin{equation}
(\mathbf{K}_1 + i k \mathbf{K}_2 + k^2 \mathbf{K}_3 - \omega^2 \mathbf{M} + i \omega \mathbf{P}) \hat{\mathbf{U}} = \hat{\mathbf{F}} \label{eigen_equation}
\end{equation}
где матрицы $\mathbf{K}_1$, $\mathbf{K}_2$, $\mathbf{K}_3$, $\mathbf{M}$, $\mathbf{P}$ формируются из значений объёмных и поверхностных интегралов в уравнениях \eqref{var_eq_solid} и \eqref{var_eq_fluid} на элементах.

Для каждого заданного значения частоты $\omega$ задача \eqref{eigen_equation} может быть сведена к обобщённой задаче на собственные значения, решением которой являются пары собственных значений и векторов $[k_m, \hat{\mathbf{U}}_m]$, соответствующие различным волновым модам системы. При наличии эффективного алгоритма отбора собственных векторов, становится возможен структурный анализ компонент волнового поля, а также их поляризации в зависимости от частоты. Результаты такого анализа в сравнении с данными обработки классическими алгоритмами приведены в разделе \ref{safe_comparison}.

\section{Результаты обработки Alford rotation}

Для демонстрации влияния несимметричности формы скважины на результаты работы алгоритмов определения главных направлений анизотропного тела были рассмотрены скважины эллиптического сечения в двух породах: Bakken Shale и Cotton Valey Shale, имеющих ТИ тип симметрии. Обе породы относятся к классу глинистых сланцев и имеют скорость распространения поперечных волн превышающую скорость звука в скважиной жидкости (так называемые быстрые породы). Трансверсально-изотропные породы описываются 5 независимыми упругими постоянными, значения которых приведены в таблице \ref{tab:properties}. Ось симметрии породы наклонена по отношению к оси скважины под углом 90\textdegree (HTI). В плоскости поперечного сечения скважины ось повернута относительно осей эллипса на $\phi = 45$\textdegree, определение этого угла и является задачей алгоритма. В расчетах использовались скважины размерами полуосей $12.70 \times 10.16$ см ($5 \times 4$ дюймов) и $15 \times 10$ см с соотношением полуосей эллипса соответственно 25\% и 50\%. Полное описание рассматриваемых моделей можно найти в таблице \ref{tab:models_description}. В качестве исходного сигнала для акустического источника в скважине использовалась производная вейвлета Блэкмана-Харриса с несущей частотой 4 кГц.

%Параметры быстрой породы, $\rho = 2640$ кг/м$^3$, $V_P = 5219$ м/с, $V_S = 3257$ м/с.
\begin{table}[H]
\footnotesize
\centering
\caption{Параметры упругих анизотропных материалов}
\renewcommand{\arraystretch}{1.5}
\begin{tabularx}{\textwidth}{|C|c|c|c|c|c|c|c|}
\hline \multirow{2}{*}{Название}  & Плотность & \multicolumn{6}{c|}{Упругие модули, ГПа} \\ 
\cline{3-8}  & кг/м$^3$ & $C_{11}$ & $C_{12}$ & $C_{13}$ & $C_{33}$ & $C_{44}$ & $C_{66}$ \\ \hline
\hline Cotton Valey Shale & 2640 & 74.73 & 14.75 & 25.29 & 58.84 & 22.05 & 29.99 \\ 
\hline Bakken Shale & 2230 & 40.9 & 10.3 & 8.5 & 26.9 & 10.5 & 15.3 \\ 
\hline 
\end{tabularx} 
\label{tab:properties}
\renewcommand{\arraystretch}{1.0}
\end{table}

\begin{table}[h]
\footnotesize
\centering
\caption{Параметры модельных задач}
\renewcommand{\arraystretch}{1.5}
\begin{tabularx}{\textwidth}{|C|l|c|l|c|c|}
\hline  Обозначение & Форма скважины & Геометрия, см & Материал породы & ТИ угол $\zeta$ & ТИ угол $\phi$ \\ \hline
\hline \textbf{}\modelnum{\label{mnum: 11}}& Эллиптическая & $12.70 \times 10.16$ & Bakken Shale & 90 & 45 \\
\hline \textbf{}\modelnum{\label{mnum: 10}}& Эллиптическая & $15.00 \times 10.00$ & Bakken Shale & 90 & 45 \\
\hline \textbf{}\modelnum{\label{mnum: 12}}& Эллиптическая & $12.70 \times 10.16$ & Cotton Valey Shale & 90 & 45 \\ 
\hline \textbf{}\modelnum{\label{mnum: 9}}& Эллиптическая & $15.00 \times 10.00$ & Cotton Valey Shale & 90 & 45 \\
\hline 
\end{tabularx}
\label{tab:models_description}
\renewcommand{\arraystretch}{1.0}
\end{table}

Основываясь на выбранной модели распространения волн, показателем качества работы алгоритмов будем считать величину $E_{cr}^n/E_{t}^n$ - относительную энергию суммы недиагональных компонент полученной матрицы $\mathbf{D}$. При обработке случаев эллиптической скважины в изотропной породе и цилиндрической скважины в анизотропной породе эта относительная энергия имеет значения порядка $10^{-7}$ для обоих алгоритмов. 

Рассмотрим результаты обработки данных численного моделирования для выбранных ранее нецилиндрических моделей. Они приведены в таблице \ref{tab:std_process_results}. Как можно заметить, в недиагональных компонентах остаётся от 1 до 3\% энергии волн, при этом оба алгоритма дают близкие значения углов без значительной неортогональности. Однако несмотря на удовлетворительные качественные показатели работы алгоритмов, значения углов значительно отличаются от заданного значения 45\textdegree.

Применение модифицированного метода Прони \cite{Ekstrom1995} к исходным трассам измерений позволяет получить дисперсионные кривые для гармоник сигнала с наиболее высокой амплитудой. В моделируемых задачах они соответствуют двум главным дипольным модам (рис. \ref{fig:disp_curves_all}). Аналогичные кривые также были построены по результатам расчётов полуаналитическим методом конечных элементов SAFE и нанесены на графики для проверки точности. Опираясь на данные дисперсионных кривых, в пакете MATLAB были построены низкочастотные и высокочастотные фильтры с конечной импульсной характеристикой (FIR), применяемые для раздельного анализа поляризации волнового поля на низких и высоких частотах. Помимо частотной фильтрации, также проводились попытки использования оконных фильтров, имеющих широкое применение при обработке каротажных данных. Однако малое количество энергии на низких частотах приводит к большим погрешностям в отфильтрованных данных и недиагонализируемости матрицы измерений.  

При обработке фильтрованных данных наблюдается заметная неортогональность между направлениями поляризации дипольных мод на низких частотах и почти полностью ортогональная ориентация вдоль осей деформации формы скважины на высоких частотах. Можно заметить, что оценка угла $\phi$ полученная классическим ортогональным алгоритмом, часто оказывается ближе к заданному в модели значению, однако, как мы покажем в следующем разделе, эти оценки не соответствуют физической поляризации распространяющихся волн.

\begin{table}[h]
\footnotesize
\centering
\caption{Результаты расчетов}
\renewcommand{\arraystretch}{1.5}
\begin{tabularx}{\textwidth}{|C|rr|rr|r|rr|}
\hline	
				&\multicolumn{1}{c}{$\theta_1^o$} & \multicolumn{1}{c|}{$\theta_1^n$} & \multicolumn{1}{c}{$\theta_2^o$} & \multicolumn{1}{c|}{$\theta_2^n$} & \multicolumn{1}{c|}{$\Delta\theta^n$}& \multicolumn{1}{c}{$E_{cr}^o/E_{t}^o$} & \multicolumn{1}{c}{$E_{cr}^n/E_{t}^n$} \\ \hline
\hline Модель \ref{mnum: 11} & 15.56 & -74.44 & 14.72 & -73.92  & 1.36  & 0.0299 & 0.0298 \\
\hline Модель \ref{mnum: 10} & 8.39 & -81.61 & 8.12 & -81.56 & 0.3 & 0.0169 & 0.0168 \\
\hline Модель \ref{mnum: 9} & 1.63 & -88.37 & 1.78 & -88.42  & 0.0  & 0.0064 & 0.0064 \\	   
\hline Модель \ref{mnum: 12} & 3.26 & 3.04 & -86.74  & -86.66  & 0.3 & 0.0075 & 0.0075 \\ 	   
 	   \hline
\end{tabularx} 
\begin{flushleft}
* здесь $\theta_1^o,\theta_2^o$ и $\theta_1^n,\theta_2^n$ соответствуют результатам, полученным ортогональной и неортогональной версией алгоритма. Величина $E_{cr}/E_{t}$ обозначает отношение энергии недиагональных компонент матрицы измерений к полной энергии.
\end{flushleft}
\label{tab:std_process_results}
\renewcommand{\arraystretch}{1.0}
\end{table}

\begin{table}[h]
\footnotesize
\centering
\caption{Результаты расчетов с применением фильтров}
\renewcommand{\arraystretch}{1.5}
\begin{tabularx}{\textwidth}{|X|rr|rr|r|ll|}
\hline
				&\multicolumn{1}{c}{$\theta_1^o$} & \multicolumn{1}{c|}{$\theta_1^n$} & \multicolumn{1}{c}{$\theta_2^o$} & \multicolumn{1}{c|}{$\theta_2^n$} & \multicolumn{1}{c|}{$\Delta\theta^n$}& \multicolumn{1}{c}{$E_{cr}^o/E_{t}^o$} & \multicolumn{1}{c}{$E_{cr}^n/E_{t}^n$} \\ \hline
\hline	\textbf{Модель \ref{mnum: 11}} & \textbf{15.56} & \textbf{14.72} & \textbf{-74.44}  & \textbf{-73.92}  & \textbf{1.36}  & \textbf{0.0299} & \textbf{0.0297} \\
		Модель \ref{mnum: 11} с НЧФ \ref{lffnum: 3} & 49.41 & 40.46 & -40.59 & -35.41  & 14.1 & 0.0204 & 0.0124\\
		Модель \ref{mnum: 11} с ВЧФ \ref{hffnum: 1} & 14.01 & 13.25 & -75.99 & -75.58  & 1.2 & 0.0039 & 0.0038\\
\hline	\textbf{Модель \ref{mnum: 10}} & \textbf{8.39} & \textbf{8.12} & \textbf{-81.61}  & \textbf{-81.56} & \textbf{0.3}  & \textbf{0.0169} & \textbf{0.0169} \\
		Модель \ref{mnum: 10} с НЧФ \ref{lffnum: 3} & 41.21 & 25.71 & -48.79 & -31.97  & 32.3 & 0.2210 & 0.1199\\
		Модель \ref{mnum: 10} с ВЧФ \ref{hffnum: 2} & 7.54 & 10.22 & -82.46 & -82.63  & 2.84 & 0.0110 & 0.0101\\
\hline	\textbf{Модель \ref{mnum: 12}} & \textbf{3.26} & \textbf{3.04} & \textbf{-86.74}  & \textbf{-86.66}  & \textbf{0.3}  & \textbf{0.0075} & \textbf{0.0075}\\
		Модель \ref{mnum: 12} с НЧФ \ref{lffnum: 4} & 48.42 & 39.73 & -41.58 & -35.50  & 14.8  & 0.0974 & 0.0701 \\
		Модель \ref{mnum: 12} с ВЧФ \ref{hffnum: 3} & 2.77 & 3.26 & -87.23 & -87.43  & 0.69  & 0.0045 & 0.0045\\	
\hline	\textbf{Модель \ref{mnum: 9}} & \textbf{1.63} & \textbf{1.78} & \textbf{-88.37}  & \textbf{-88.42}  & \textbf{0.0}  & \textbf{0.0064} & \textbf{0.0064} \\
		Модель \ref{mnum: 9} с НЧФ \ref{lffnum: 4} & 6.03 & 7.73 & -83.97 & -56.60  & 25.6  & 0.0789 & 0.0726 \\
		Модель \ref{mnum: 9} с ВЧФ \ref{hffnum: 3} & 1.49 & 2.04 & -88.51 & -88.71  & 0.74  & 0.0184 & 0.0183\\		
\hline	
\end{tabularx} 
\begin{flushleft}
* здесь $\theta_1^o,\theta_2^o$ и $\theta_1^n,\theta_2^n$ соответствуют результатам, полученным ортогональной и неортогональной версией алгоритма. Величина $E_{cr}/E_{t}$ обозначает отношение энергии недиагональных компонент матрицы измерений к полной энергии.
\end{flushleft}
\label{tab:filter_process_results}
\renewcommand{\arraystretch}{1.0}
\end{table}


\section{Сравнение с SAFE}
\label{safe_comparison}

Структура решения, полученного методом SAFE, позволяет выделять и рассматривать отдельные компоненты волнового поля, анализировать их поляризацию в зависимости от частоты. На рисунках \ref{fig:bs10_8_HTI45} и \ref{fig:bs15_10_HTI45} приведена визуализация амплитуды давления собственных векторов, относящихся к двум дипольным модам внутри скважины для моделей \ref{mnum: 11} и \ref{mnum: 10}. Направление градиента отражает направление поляризации каждой из волн, которое меняется в зависимости от частоты. Отметим также, что поляризация на низких частотах практически не отличается для случаев с эллиптичностью 25\% и 50\%. Наибольшая неортогональность этих направлений наблюдается в диапазоне частот от 2 до 4 кГц для рассматриваемых случаев. 

Для лучшего восприятия данных моделирования на полученные профили амплитуд нанесены результаты обработки данных из таблицы \ref{tab:filter_process_results}: для частот 1.94 кГц и 3.29 кГц нанесены результаты с применением низкочастотной фильтрации и без, для частот 5.52 кГц и 8.65 кГц - высокочастотной фильтрации и без неё. 

\begin{figure}[h]
\centering
\begin{minipage}{0.49\linewidth}
	\centering \textbf{Модель \ref{mnum: 11}}
	\psfragfig[width=0.49\linewidth,crop=pdfcrop]{./images/nonorth_alford/el10x8_inch_HTI_BS_f45_disp_modes+SAFE} \\
\end{minipage}
\begin{minipage}{0.49\linewidth}
	\centering \textbf{Модель \ref{mnum: 10}}
	\psfragfig[width=0.49\linewidth,crop=pdfcrop]{./images/nonorth_alford/el15x10_HTI_BS_f45_disp_modes+SAFE} \\
\end{minipage}
\begin{minipage}{0.49\linewidth}
	\centering \textbf{Модель \ref{mnum: 12}}
	\psfragfig[width=0.49\linewidth,crop=pdfcrop]{./images/nonorth_alford/el10x8_inch_HTI_CS_f45_disp_modes+SAFE} \\
\end{minipage}
\begin{minipage}{0.49\linewidth}
	\centering \textbf{Модель \ref{mnum: 9}}
	\psfragfig[width=0.49\linewidth,crop=pdfcrop]{./images/nonorth_alford/el15x10_HTI_CS_f45_disp_modes+SAFE_new} \\
\end{minipage}
\caption{Дисперсионные кривые для основных задач. }
\label{fig:disp_curves_all}
\end{figure}

\begin{figure}[h]
\centering
\begin{tabular*}{\textwidth}{@{\extracolsep{\fill} }crl}
& \textit{Низкочастотная фильтрация} & \textit{Высокочастотная фильтрация}\\
\end{tabular*}
\begin{minipage}{0.04\linewidth}
\rotatebox{90}{\textit{Дипольная мода 1}} 
\end{minipage}
\begin{minipage}{0.23\linewidth}
	\psfragfig[width=0.24\linewidth,crop=pdfcrop]{./images/SAFE/SAFE_BS_10x8_HTI_45/P_s_2kHz}		
\end{minipage}
\begin{minipage}{0.23\linewidth}
	\psfragfig[width=0.24\linewidth,crop=pdfcrop]{./images/SAFE/SAFE_BS_10x8_HTI_45/P_s_3_3kHz}		
\end{minipage}
\begin{minipage}{0.23\linewidth}
	\psfragfig[width=0.24\linewidth,crop=pdfcrop]{./images/SAFE/SAFE_BS_10x8_HTI_45/P_s_5_5kHz}		
\end{minipage}
\begin{minipage}{0.23\linewidth}
	\psfragfig[width=0.24\linewidth,crop=pdfcrop]{./images/SAFE/SAFE_BS_10x8_HTI_45/P_s_8_6kHz}		
\end{minipage}
\begin{minipage}{0.04\linewidth}
\rotatebox{90}{\textit{Дипольная мода 2}} 
\end{minipage}
\begin{minipage}{0.23\linewidth}
	\psfragfig[width=0.24\linewidth,crop=pdfcrop]{./images/SAFE/SAFE_BS_10x8_HTI_45/P_a_2kHz}		
\end{minipage}
\begin{minipage}{0.23\linewidth}
	\psfragfig[width=0.24\linewidth,crop=pdfcrop]{./images/SAFE/SAFE_BS_10x8_HTI_45/P_a_3_3kHz}		
\end{minipage}
\begin{minipage}{0.23\linewidth}
	\psfragfig[width=0.24\linewidth,crop=pdfcrop]{./images/SAFE/SAFE_BS_10x8_HTI_45/P_a_5_5kHz}		
\end{minipage}
\begin{minipage}{0.23\linewidth}
	\psfragfig[width=0.24\linewidth,crop=pdfcrop]{./images/SAFE/SAFE_BS_10x8_HTI_45/P_a_8_6kHz}		
\end{minipage}
\vspace{\baselineskip}
\renewcommand{\arraystretch}{1.0}
\footnotesize
\begin{tabular*}{\textwidth}{@{\extracolsep{\fill} }clclcl}
\tikz \draw (0,0) -- (1cm,0); & ортогональный & \tikz \draw[dashed] (0,0) -- (1cm,0); & ортогональный & \tikz \draw[dashdotted] (0,0) -- (1cm,0); & неортогональный    \\
& без фильтрации & & с фильтрацией & & с фильтрацией \\
\end{tabular*}
\renewcommand{\arraystretch}{1.0}
\normalsize
\caption{Результаты расчетов собственных векторов для Модели \ref{mnum: 11}}
\label{fig:bs10_8_HTI45}
\end{figure}

\begin{figure}[h]
\centering
\begin{tabular*}{\textwidth}{@{\extracolsep{\fill} }crl}
& \textit{Низкочастотная фильтрация} & \textit{Высокочастотная фильтрация}\\
\end{tabular*}
\begin{minipage}{0.04\linewidth}
\rotatebox{90}{\textit{Дипольная мода 1}} 
\end{minipage}
\begin{minipage}{0.23\linewidth}
	\psfragfig[width=0.24\linewidth,crop=pdfcrop]{./images/SAFE/SAFE_BS_15x10_HTI_45/P_s_2kHz}		
\end{minipage}
\begin{minipage}{0.23\linewidth}
	\psfragfig[width=0.24\linewidth,crop=pdfcrop]{./images/SAFE/SAFE_BS_15x10_HTI_45/P_s_3_3kHz}		
\end{minipage}
\begin{minipage}{0.23\linewidth}
	\psfragfig[width=0.24\linewidth,crop=pdfcrop]{./images/SAFE/SAFE_BS_15x10_HTI_45/P_s_5_5kHz}		
\end{minipage}
\begin{minipage}{0.23\linewidth}
	\psfragfig[width=0.24\linewidth,crop=pdfcrop]{./images/SAFE/SAFE_BS_15x10_HTI_45/P_s_8_6kHz}		
\end{minipage}
\begin{minipage}{0.04\linewidth}
\rotatebox{90}{\textit{Дипольная мода 2}} 
\end{minipage}
\begin{minipage}{0.23\linewidth}
	\psfragfig[width=0.24\linewidth,crop=pdfcrop]{./images/SAFE/SAFE_BS_15x10_HTI_45/P_a_2kHz}		
\end{minipage}
\begin{minipage}{0.23\linewidth}
	\psfragfig[width=0.24\linewidth,crop=pdfcrop]{./images/SAFE/SAFE_BS_15x10_HTI_45/P_a_3_3kHz}		
\end{minipage}
\begin{minipage}{0.23\linewidth}
	\psfragfig[width=0.24\linewidth,crop=pdfcrop]{./images/SAFE/SAFE_BS_15x10_HTI_45/P_a_5_5kHz}		
\end{minipage}
\begin{minipage}{0.23\linewidth}
	\psfragfig[width=0.24\linewidth,crop=pdfcrop]{./images/SAFE/SAFE_BS_15x10_HTI_45/P_a_8_6kHz}		
\end{minipage}
\vspace{\baselineskip}
\renewcommand{\arraystretch}{1.0}
\footnotesize
\begin{tabular*}{\textwidth}{@{\extracolsep{\fill} }clclcl}
\tikz \draw (0,0) -- (1cm,0); & ортогональный & \tikz \draw[dashed] (0,0) -- (1cm,0); & ортогональный & \tikz \draw[dashdotted] (0,0) -- (1cm,0); & неортогональный    \\
& без фильтрации & & с фильтрацией & & с фильтрацией \\
\end{tabular*}
\renewcommand{\arraystretch}{1.0}
\normalsize
\caption{Результаты расчетов собственных векторов для Модели \ref{mnum: 10}}
\label{fig:bs15_10_HTI45}
\end{figure}

\begin{figure}[h]
\centering
\begin{tabular*}{\textwidth}{@{\extracolsep{\fill} }crl}
& \textit{Низкочастотная фильтрация} & \textit{Высокочастотная фильтрация}\\
\end{tabular*}
\begin{minipage}{0.04\linewidth}
\rotatebox{90}{\textit{Дипольная мода 1}} 
\end{minipage}
\begin{minipage}{0.23\linewidth}
	\psfragfig[width=0.24\linewidth,crop=pdfcrop]{./images/SAFE/SAFE_CS_15x10_HTI_45/P_s_2kHz}		
\end{minipage}
\begin{minipage}{0.23\linewidth}
	\psfragfig[width=0.24\linewidth,crop=pdfcrop]{./images/SAFE/SAFE_CS_15x10_HTI_45/P_s_3_3kHz}		
\end{minipage}
\begin{minipage}{0.23\linewidth}
	\psfragfig[width=0.24\linewidth,crop=pdfcrop]{./images/SAFE/SAFE_CS_15x10_HTI_45/P_s_5_5kHz}		
\end{minipage}
\begin{minipage}{0.23\linewidth}
	\psfragfig[width=0.24\linewidth,crop=pdfcrop]{./images/SAFE/SAFE_CS_15x10_HTI_45/P_s_8_6kHz}		
\end{minipage}
\begin{minipage}{0.04\linewidth}
\rotatebox{90}{\textit{Дипольная мода 2}} 
\end{minipage}
\begin{minipage}{0.23\linewidth}
	\psfragfig[width=0.24\linewidth,crop=pdfcrop]{./images/SAFE/SAFE_CS_15x10_HTI_45/P_a_2kHz}		
\end{minipage}
\begin{minipage}{0.23\linewidth}
	\psfragfig[width=0.24\linewidth,crop=pdfcrop]{./images/SAFE/SAFE_CS_15x10_HTI_45/P_a_3_3kHz}		
\end{minipage}
\begin{minipage}{0.23\linewidth}
	\psfragfig[width=0.24\linewidth,crop=pdfcrop]{./images/SAFE/SAFE_CS_15x10_HTI_45/P_a_5_5kHz}		
\end{minipage}
\begin{minipage}{0.23\linewidth}
	\psfragfig[width=0.24\linewidth,crop=pdfcrop]{./images/SAFE/SAFE_CS_15x10_HTI_45/P_a_8_6kHz}		
\end{minipage}
\vspace{\baselineskip}
\renewcommand{\arraystretch}{1.0}
\footnotesize
\begin{tabular*}{\textwidth}{@{\extracolsep{\fill} }clclcl}
\tikz \draw (0,0) -- (1cm,0); & ортогональный & \tikz \draw[dashed] (0,0) -- (1cm,0); & ортогональный & \tikz \draw[dashdotted] (0,0) -- (1cm,0); & неортогональный    \\
& без фильтрации & & с фильтрацией & & с фильтрацией \\
\end{tabular*}
\renewcommand{\arraystretch}{1.0}
\normalsize
\caption{Результаты расчетов собственных векторов для Модели \ref{mnum: 9}}
\label{fig:cs15_10_HTI45}
\end{figure}

Спектр фильтрованного сигнала достаточно широк и оба алгоритма дают оценку некоторого среднего направления ориентации мод. Представленные данные хорошо демонстрируют, что неортогональная версия Alford rotation дает более близкие оценки направлений поляризации дипольных волн на низких частотах. Как можно заметить, направления поляризации основных мод не совпадают с заданной ориентацией оси симметрии трансверсально-изотропной породы. Близкие к 45\textdegree значения угла классического алгоритма вероятно являются лишь случайным совпадением осредненных ортогональным способом реальных поляризаций мод на этих частотах с заданным значением в модели. В пользу последнего утверждения также говорит и тот факт, что энергия недиагональных компонент при ортогональной обработке почти на 10\% выше. 

Результаты обработки нефильтрованного сигнала в рассмотренных задачах, как видно из данных таблиц, полностью определяются ориентацией мод на высоких частотах. Интересно, что при этом поляризация мод почти ортогональна и может не совпадать геометрически с ориентацией полуосей эллипса поперечного сечения скважины (см. модель \ref{mnum: 11}-\ref{mnum: 10}), как в случае с изотропной породой. При увеличении степени эллиптичности ствола это различие сокращается. Таким образом, даже при корректной (с точки зрения диагонализации матрицы измерений) работе алгоритма полученное значение угла на направление главной оси анизотропного материала может не отвечать ни физическим свойствам породы, ни геометрии задачи. 
\clearpage

\section{Заключение}

Как следует из данного исследования, результаты обработки каротажных измерений в быстрых породах ортогональными и неортогональными методами, основанными на диагонализации матрицы измерений, в значительной части определяются поляризацией нормальных мод на высоких частотах. При этом на примере расчета задачи нецилиндрической скважины в породе Bakken shale показано, что эти результаты могут не соответствовать главным направлениям анизотропной породы, что является серьезным недостатком при оценке этого параметра в задачах интерпретации.

Применение частотной фильтрации позволяет получить более точные оценки главных направлений анизотропной породы, однако требует учёта возможной неортогональности этих направлений. При исследовании пород с высокой степенью симметрии тензора упругих постоянных, ортогональность на низких частотах можно считать маркером корректности решения задачи. 

%Важным результатом данной работы является оценка точности работы упомянутых алгоритмов, а также демонстрация возможного существования ортогонально поляризованных волн в быстрых породах,  не связанных однозначно с одним из факторов. Для дипольных мод на средних частотах существует область с достаточно резкой сменой направлений поляризации, которая однако не имеет определяющего значения для работы алгоритма, так как спектр используемых источников значительно шире.

Приведённый в статье материал демонстрирует возможности спектральных методов, схожих с полуаналитическим методом конечных элементов (SAFE), по анализу и интерпретации отдельных аспектов волнового поля в скважинах. При наличии данных о геометрии и неортогональности главных направлений изгибных волн, данный метод может быть использован для коррекции результатов обработки каротажных измерений и решения обратных задач инверсии параметров упругой среды.
%\section{Краткое описание проблемы}

Модель распространения волн по скважине, используемой в алгоритмах обработки данных каротажных измерений, основана на уравнениях распространения плоской волны в анизотропной недисперсионной среде. Вектор значений $\mathbf{R}$ на приемниках может быть выражен через вектор возбуждения источника $\mathbf{S}$ в форме
$$
	\mathbf{R} = \mathbf{P}_{M \rightarrow R} \ \mathbf{D} \ \mathbf{P}_{S \rightarrow M} \ \mathbf{S},
$$
где $\mathbf{P}_{S \rightarrow M}$ - матрица, проецирующая вектор источника на главные направления распространения нормальных дипольных мод, $\mathbf{D}$ - матрица, определяющая распространение чистых мод вдоль скважины (в предположении, что моды не взаимодействуют друг с другом, считаем $\mathbf{D}$ диагональной), $\mathbf{P}_{M \rightarrow R}$ - матрица, проецирующая сигнал чистых мод на направления приемников.

В типовой схеме кросс-дипольных измерений с двумя ортогональными источниками, сонаправленными с осями $X$ и $Y$, $\mathbf{S}$ представляется через единичную матрицу, а данные с приемников могут быть записаны в форме матрицы данных:
$$
\left(
\begin{array}{cc}
XX & YX \\
XY & YY \\
\end{array}
\right) = \mathbf{R} = \mathbf{P}_{M \rightarrow R} \ \mathbf{D} \ \mathbf{P}_{S \rightarrow M}
$$
Если системы координат источников и приемников совпадают, то $\mathbf{P}_{M \rightarrow R}={\mathbf{P}_{S \rightarrow M}}^{T} = \mathbf{P}$, а матрица распространения чистых мод может быть выражена в форме
$$
	\mathbf{D} = \mathbf{P}^{-1} \ \mathbf{R} \ \mathbf{P}^{-T}.
$$
В традиционном варианте Alford rotation \cite{Alford1986} моды считаются ортогональными. В этом случае ортогональным поворотом на угол $\theta$ можно перейти в каноническую систему координат. В работе Dellinger et al. \cite{Dellinger1998} было показано, что даже если моды в скважине не являются ортогональными, $\mathbf{D}$ все равно может быть диагонализирована. Направление поляризации первой моды характеризуется углом $\theta$ относительно оси $X$, направление второй моды определяется поворотом на $\theta + \eta$ относительно оси $Y$.
\\

\begin{parcolumns}[colwidths={1=0.5\linewidth},rulebetween]{2}

\colchunk{
\textbf{Ортогональный Alford rotation}
\begin{align*}
\mathbf{P} & = \left(
\begin{array}{cc}
\cos \theta & -\sin \theta \\ 
\sin \theta & \cos \theta
\end{array} 
\right) \\
\mathbf{P}^{-1} & = \left(
\begin{array}{cc}
\cos \theta & \sin \theta \\ 
-\sin \theta & \cos \theta
\end{array} 
\right)
\end{align*}
}
\colchunk{
\textbf{Неортогональный Alford rotation}
\begin{align*}
\mathbf{P} &= \left(
\begin{array}{cc}
\cos \theta & -\sin (\theta+\eta) \\ 
\sin \theta & \cos (\theta+\eta)
\end{array} 
\right) \\
\mathbf{P}^{-1} &= \frac{1}{\cos \eta} \left(
\begin{array}{cc}
\cos (\theta+\eta) & \sin (\theta+\eta) \\ 
-\sin \theta & \cos (\theta)
\end{array} 
\right)
\end{align*}
}
\colplacechunks
\end{parcolumns}
%\begin{figure}[h]
\centering
\begin{minipage}{0.24\linewidth}
	\psfragfig[width=0.24\linewidth,crop=pdfcrop]{./images/SAFE/CS_15x10/cs15x10_sym_f1_5kHz+ang}		
\end{minipage}
\begin{minipage}{0.24\linewidth}
	\psfragfig[width=0.24\linewidth,crop=pdfcrop]{./images/SAFE/CS_15x10/cs15x10_sym_f2_39kHz+ang}		
\end{minipage}
\begin{minipage}{0.24\linewidth}
	\psfragfig[width=0.24\linewidth,crop=pdfcrop]{./images/SAFE/CS_15x10/cs15x10_sym_f3_28kHz+ang}		
\end{minipage}
\begin{minipage}{0.24\linewidth}
	\psfragfig[width=0.24\linewidth,crop=pdfcrop]{./images/SAFE/CS_15x10/cs15x10_sym_f5_52kHz+ang}		
\end{minipage}
\hfill
\begin{minipage}{0.24\linewidth}
	\psfragfig[width=0.24\linewidth,crop=pdfcrop]{./images/SAFE/CS_15x10/cs15x10_asym_f1_5kHz+ang}		
\end{minipage}
\begin{minipage}{0.24\linewidth}
	\psfragfig[width=0.24\linewidth,crop=pdfcrop]{./images/SAFE/CS_15x10/cs15x10_asym_f2_39kHz+ang}		
\end{minipage}
\begin{minipage}{0.24\linewidth}
	\psfragfig[width=0.24\linewidth,crop=pdfcrop]{./images/SAFE/CS_15x10/cs15x10_asym_f3_28kHz+ang}		
\end{minipage}
\begin{minipage}{0.24\linewidth}
	\psfragfig[width=0.24\linewidth,crop=pdfcrop]{./images/SAFE/CS_15x10/cs15x10_asym_f5_52kHz+ang}		
\end{minipage}
\hfill
\vspace{\baselineskip}
\begin{minipage}{0.24\linewidth}
	\psfragfig[width=0.24\linewidth,crop=pdfcrop]{./images/SAFE/CS_15x10/p_cs15x10_sym_f1_5kHz+ang}		
\end{minipage}
\begin{minipage}{0.24\linewidth}
	\psfragfig[width=0.24\linewidth,crop=pdfcrop]{./images/SAFE/CS_15x10/p_cs15x10_sym_f2_39kHz+ang}		
\end{minipage}
\begin{minipage}{0.24\linewidth}
	\psfragfig[width=0.24\linewidth,crop=pdfcrop]{./images/SAFE/CS_15x10/p_cs15x10_sym_f3_28kHz+ang}		
\end{minipage}
\begin{minipage}{0.24\linewidth}
	\psfragfig[width=0.24\linewidth,crop=pdfcrop]{./images/SAFE/CS_15x10/p_cs15x10_sym_f5_52kHz+ang}		
\end{minipage}
\hfill
\begin{minipage}{0.24\linewidth}
	\psfragfig[width=0.24\linewidth,crop=pdfcrop]{./images/SAFE/CS_15x10/p_cs15x10_asym_f1_5kHz+ang}		
\end{minipage}
\begin{minipage}{0.24\linewidth}
	\psfragfig[width=0.24\linewidth,crop=pdfcrop]{./images/SAFE/CS_15x10/p_cs15x10_asym_f2_39kHz+ang}		
\end{minipage}
\begin{minipage}{0.24\linewidth}
	\psfragfig[width=0.24\linewidth,crop=pdfcrop]{./images/SAFE/CS_15x10/p_cs15x10_asym_f3_28kHz+ang}		
\end{minipage}
\begin{minipage}{0.24\linewidth}
	\psfragfig[width=0.24\linewidth,crop=pdfcrop]{./images/SAFE/CS_15x10/p_cs15x10_asym_f5_52kHz+ang}		
\end{minipage}
\hfill

\caption{Графическое представление собственных векторов дипольных мод внутри (поле давления) и снаружи (компонента смещения z) скважины. Модель \ref{mnum: 5}. Здесь сплошные линии указывают направления, полученные неортогональным Alford rotation, примененным к исходным данным; прерывистая линия - к данным с оконной фильтрацией, прерывистая с точкой - к данным с низкочастотной фильтрацией.}
\end{figure}

\begin{figure}[h]
\centering
	\psfragfig[width=1\linewidth,crop=pdfcrop]{./images/SAFE/CS_15x10/modes_U+angles}		
	\caption{Представление значения $U = \sqrt{U_x^2+U_y^2+U_z^2}$ для собственных векторов в породе в зависимости от частоты. Модель \ref{mnum: 5}, быстрая порода. }
\end{figure}

\begin{figure}[h]
\centering
\begin{minipage}{0.24\linewidth}
	\psfragfig[width=0.24\linewidth,crop=pdfcrop]{./images/SAFE/AC_15x10/ac15x10_sym_f1_5kHz+ang}		
\end{minipage}
\begin{minipage}{0.24\linewidth}
	\psfragfig[width=0.24\linewidth,crop=pdfcrop]{./images/SAFE/AC_15x10/ac15x10_sym_f3_28kHz+ang}		
\end{minipage}
\begin{minipage}{0.24\linewidth}
	\psfragfig[width=0.24\linewidth,crop=pdfcrop]{./images/SAFE/AC_15x10/ac15x10_sym_f5_52kHz+ang}		
\end{minipage}
\begin{minipage}{0.24\linewidth}
	\psfragfig[width=0.24\linewidth,crop=pdfcrop]{./images/SAFE/AC_15x10/ac15x10_sym_f10_0kHz+ang}		
\end{minipage}
\hfill
\begin{minipage}{0.24\linewidth}
	\psfragfig[width=0.24\linewidth,crop=pdfcrop]{./images/SAFE/AC_15x10/ac15x10_asym_f1_5kHz+ang}		
\end{minipage}
\begin{minipage}{0.24\linewidth}
	\psfragfig[width=0.24\linewidth,crop=pdfcrop]{./images/SAFE/AC_15x10/ac15x10_asym_f3_28kHz+ang}		
\end{minipage}
\begin{minipage}{0.24\linewidth}
	\psfragfig[width=0.24\linewidth,crop=pdfcrop]{./images/SAFE/AC_15x10/ac15x10_asym_f5_52kHz+ang}		
\end{minipage}
\begin{minipage}{0.24\linewidth}
	\psfragfig[width=0.24\linewidth,crop=pdfcrop]{./images/SAFE/AC_15x10/ac15x10_asym_f10_0kHz+ang}		
\end{minipage}
\hfill
\vspace{\baselineskip}
\begin{minipage}{0.24\linewidth}
	\psfragfig[width=0.24\linewidth,crop=pdfcrop]{./images/SAFE/AC_15x10/p_ac15x10_sym_f1_5kHz+ang}		
\end{minipage}
\begin{minipage}{0.24\linewidth}
	\psfragfig[width=0.24\linewidth,crop=pdfcrop]{./images/SAFE/AC_15x10/p_ac15x10_sym_f3_28kHz+ang}		
\end{minipage}
\begin{minipage}{0.24\linewidth}
	\psfragfig[width=0.24\linewidth,crop=pdfcrop]{./images/SAFE/AC_15x10/p_ac15x10_sym_f5_52kHz+ang}		
\end{minipage}
\begin{minipage}{0.24\linewidth}
	\psfragfig[width=0.24\linewidth,crop=pdfcrop]{./images/SAFE/AC_15x10/p_ac15x10_sym_f10_0kHz+ang}		
\end{minipage}
\hfill
\begin{minipage}{0.24\linewidth}
	\psfragfig[width=0.24\linewidth,crop=pdfcrop]{./images/SAFE/AC_15x10/p_ac15x10_asym_f1_5kHz+ang}		
\end{minipage}
\begin{minipage}{0.24\linewidth}
	\psfragfig[width=0.24\linewidth,crop=pdfcrop]{./images/SAFE/AC_15x10/p_ac15x10_asym_f3_28kHz+ang}		
\end{minipage}
\begin{minipage}{0.24\linewidth}
	\psfragfig[width=0.24\linewidth,crop=pdfcrop]{./images/SAFE/AC_15x10/p_ac15x10_asym_f5_52kHz+ang}		
\end{minipage}
\begin{minipage}{0.24\linewidth}
	\psfragfig[width=0.24\linewidth,crop=pdfcrop]{./images/SAFE/AC_15x10/p_ac15x10_asym_f10_0kHz+ang}		
\end{minipage}
\hfill

\caption{Графическое представление собственных векторов дипольных мод внутри (поле давления) и снаружи (компонента смещения z) скважины. Модель \ref{mnum: 3}. Здесь сплошные линии указывают направления, полученные неортогональным Alford rotation, примененным к исходным данным; прерывистая линия - к данным с низкочастотной фильтрацией, прерывистая с точкой - к данным с высокочастотной фильтрацией.}
\end{figure}

\begin{figure}[h]
\centering
	\psfragfig[width=1\linewidth,crop=pdfcrop]{./images/SAFE/AC_15x10/ac15x10_modes_U+angles}
	\caption{Представление значения $U = \sqrt{U_x^2+U_y^2+U_z^2}$ для собственных векторов в породе в зависимости от частоты. Модель \ref{mnum: 3}, медленная порода. }		
\end{figure}

\textbf{Зависимость энергии диагональных компонент от значений $\theta$ и $\eta$}\\
			\psfragfig[width=0.40\linewidth,crop=pdfcrop]{./images/nonorth_alford/solution_min_gs_rot4c}
			\label{fig:rot4_gs_solution}

\begin{minipage}[h]{0.47\linewidth}
\begin{center}
\textbf{Результат работы TKO}
			\psfragfig[width=0.40\linewidth,crop=pdfcrop]{./images/nonorth_alford/el20x10_TTI60_TKO_compare}\\
	  		\label{fig:rot4_tko_comp}
\end{center}	  		
\end{minipage}

\begin{figure}[h]
\centering
\begin{minipage}{0.47\linewidth}
	\psfragfig[width=0.47\linewidth,crop=pdfcrop]{./images/SAFE/CS_15x10/cs15x10_sym_f1_5kHz_U+ang}		
\end{minipage}
\hfill
\begin{minipage}{0.47\linewidth}
	\psfragfig[width=0.47\linewidth,crop=pdfcrop]{./images/SAFE/CS_15x10/cs15x10_sym_f7_76kHz_U+ang}		
\end{minipage}
\vspace{\baselineskip}
\begin{minipage}{0.47\linewidth}
	\psfragfig[width=0.47\linewidth,crop=pdfcrop]{./images/SAFE/CS_15x10/cs15x10_asym_f1_5kHz_U+ang}		
\end{minipage}
\hfill
\begin{minipage}{0.47\linewidth}
	\psfragfig[width=0.47\linewidth,crop=pdfcrop]{./images/SAFE/CS_15x10/cs15x10_asym_f7_76kHz_U+ang}		
\end{minipage}
\caption{Представление значения $|U|$ для собственных векторов в породе в зависимости от частоты. Модель \ref{mnum: 5}, быстрая порода. Здесь сплошные линии указывают направления, полученные неортогональным Alford rotation, примененным к исходным данным; прерывистая линия - к данным с оконной фильтрацией, прерывистая с точкой - к данным с низкочастотной фильтрацией. }	
\end{figure}

\begin{figure}[h]
\begin{minipage}{1\linewidth}
	\psfragfig[width=1\linewidth,crop=pdfcrop]{./images/SAFE/CS_15x10/disp_close_spec}		
\end{minipage}
\vspace{\baselineskip}
\begin{minipage}{1\linewidth}
	\psfragfig[width=1\linewidth,crop=pdfcrop]{./images/SAFE/CS_15x10/modes_close_spec}		
\end{minipage}
\end{figure}


%\bibliography{./library/library}
%\include{include/var_bibliography_link}
%\bibliography{c:/Users/German/Documents/TeX_Library/library}
\bibliography{d:/Documents/Workfiles/Literature/TeX_Library/library}

\bibliographystyle{unsrt}
%\bibliographystyle{plainnat_no_url}

\appendix
\section{Фильтрация данных}

%\begin{minipage}[c]{0.47\linewidth}	
%\begin{center}
%		Ортогональный Alford rotation \\
%		\psfragfig[width=0.40\linewidth,crop=pdfcrop]{./images/nonorth_alford/el20x10_TTI60_rot4c_scheme}			
%	  		\label{fig:rot4_scheme}
%\end{center}	  		
%\end{minipage} \hfill
%\begin{minipage}[c]{0.47\linewidth}
%\begin{center}
%		Неортогональный Alford rotation\\
%			\psfragfig[width=0.40\linewidth,crop=pdfcrop]{./images/nonorth_alford/el20x10_TTI60_gs_rot4c_scheme}
%			\label{fig:rot4_gs_scheme}
%\end{center}
%\end{minipage} 	\\
\begin{table}[H]
\footnotesize
%\centering
\caption{Параметры применяемых фильтров}
\renewcommand{\arraystretch}{1.5}
\textbf{Оконные фильтры} \\
\begin{tabularx}{\textwidth}{|C|c|c|}
\hline Обозначение & Величина окна, мс & Интервальное время, мкс/м  \\ \hline
\hline \textbf{}\wfiltnum{\label{wfnum: 1}} & 0.5 & 987  \\ 
\hline \textbf{}\wfiltnum{\label{wfnum: 2}} & 1.0 & 400  \\ 
\hline \textbf{}\wfiltnum{\label{wfnum: 3}} & 0.5 & 320  \\
\hline \textbf{}\wfiltnum{\label{wfnum: 4}} & 0.5 & 650  \\
\hline \textbf{}\wfiltnum{\label{wfnum: 5}} & 0.6 & 320  \\
\hline \textbf{}\wfiltnum{\label{wfnum: 6}} & 0.5 & 400  \\
\hline 
\end{tabularx} \\
\textbf{Низкочастотные фильтры} \\
\begin{tabularx}{\textwidth}{|C|c|c|c|c|c|}
\hline Обозначение & Частота дискретизации, Гц & $A_{pass}, $ дБ & $A_{stop}, $ дБ & $F_{pass}, $ Гц & $F_{stop}, $ Гц \\ \hline
\hline \textbf{}\lffiltnum{\label{lffnum: 1}} & 1194892 & 1 & 80 & 4000 & 5000 \\ 
\hline \textbf{}\lffiltnum{\label{lffnum: 2}} & 1194892 & 1 & 80 & 5000 & 6000 \\ 
\hline \textbf{}\lffiltnum{\label{lffnum: 3}} & - & 1 & 80 & 3000 & 4000 \\ 
\hline \textbf{}\lffiltnum{\label{lffnum: 4}} & - & 1 & 80 & 3000 & 3500 \\ 
\hline 
\end{tabularx} \\
\textbf{Высокочастотные фильтры} \\
\begin{tabularx}{\textwidth}{|C|c|c|c|c|c|}
\hline Обозначение & Частота дискретизации, Гц & $A_{pass}, $ дБ & $A_{stop}, $ дБ & $F_{pass}, $ Гц & $F_{stop}, $ Гц \\ \hline
\hline \textbf{}\hffiltnum{\label{hffnum: 1}} & 1194892 & 1 & 80 & 5000 & 4000 \\ 
\hline \textbf{}\hffiltnum{\label{hffnum: 2}} & 1194892 & 1 & 80 & 6000 & 5000 \\ 
\hline \textbf{}\hffiltnum{\label{hffnum: 3}} & - & 1 & 80 & 7000 & 6000 \\ 
\hline 
\end{tabularx}
\renewcommand{\arraystretch}{1.0}
\end{table}

\end{document} 