% !TeX spellcheck = ru_RU
\documentclass[a4paper,11pt]{article}
\usepackage[process=auto]{pstool}
\usepackage[english,russian]{babel}
\usepackage[T2A]{fontenc}
\usepackage[utf8]{inputenc}
\usepackage{amssymb,amsmath}
\usepackage{gensymb,textcomp,latexsym}
\usepackage{graphicx}
\usepackage{tabularx}
\usepackage[pdftex, left=1in, right=1in, top=1in, bottom=2cm]{geometry}
\usepackage{parcolumns}
\usepackage{multirow}
\usepackage{tikz}

%\usepackage[usenames,dvipsnames]{xcolor}
%\usepackage[font=small,labelfont=bf]{caption}
\usepackage[center]{subfigure}
\renewcommand{\thesubfigure}{(\asbuk{subfigure})~}
\newcommand{\figref}[1]{Рис.~\ref{#1}}
\usepackage[pdfauthor={Shchelik},pdftitle={Nonorthogonal Alford test},pdfstartview=XYZ,bookmarks=true,colorlinks=true,linkcolor=blue,urlcolor=blue,citecolor=blue,
bookmarks=true,linktocpage=true,hyperindex=true]{hyperref}
%\usepackage[hyperpageref]{backref}

%\usepackage[section]{placeins}
%\usepackage{graphicx}
%\usepackage{epsfig}
%\usepackage{epstopdf}
%\usepackage{subfigure}
\usepackage{float}
%\floatstyle{boxed}
%\restylefloat{figure}
%\usepackage{booktabs}
%\graphicspath{ {./image_test/} }

\newcommand{\ii}{\mathrm{i}}

\newcounter{modelnum}
\newcommand{\modelnum}[1]{\refstepcounter{modelnum}Модель \themodelnum #1}

%Filters counters
\newcounter{wfiltnum}
\newcommand{\wfiltnum}[1]{\refstepcounter{wfiltnum}ОФ-\thewfiltnum #1}
\newcounter{lffiltnum}
\newcommand{\lffiltnum}[1]{\refstepcounter{lffiltnum}НЧФ-\thelffiltnum #1}
\newcounter{hffiltnum}
\newcommand{\hffiltnum}[1]{\refstepcounter{hffiltnum}ВЧФ-\thehffiltnum #1}
%tabularx options
\newcolumntype{C}{>{\centering}X}


\begin{document}
\part*{Анализ поляризации дипольных волн в скважинах некругового сечения в анизотропной породе}
\today

\begin{abstract}

В работе исследуется вопрос	определения главных направлений анизотропной породы в скважинах с нарушением цилиндрической геометрии с помощью численного моделирования измерений акустического каротажа. Модель используемых на практике алгоритмов обработки предполагает распространение вдоль скважины двух ортогонально поляризованных волн, которые в рассматриваемых задачах соответствуют дипольным модам. На примере эллиптических скважин в работе показано, что направления колебаний мод могут быть существенно неортогональными и зависеть от частотного спектра сигнала источника, что приводит к некорректному определению главных направлений трансверсально-изотропной породы. Полученные после обработки направления сопоставлены с независимым расчётом собственных векторов дипольных мод полуаналитическим методом конечных элементов (SAFE). Результаты сравнения свидетельствуют об эффективности применения частотных фильтров и "неортогональных"\ алгоритмов для проверки корректности найденных направлений и повышения точности значений углов. Приведённые в работе результаты имеют особую ценность для обработки каротажных измерений в наклонных и горизонтальных скважинах с деформациями ствола.
\end{abstract}

\section{Введение}
Уже несколько десятилетий в акустическом каротаже широко и достаточно успешно практикуются методики кросс-дипольных измерений. Современные решения в технике и обработке полученных данных позволяют количественно производить оценку азимутальной и аксиальной (по отношению к стволу скважины) анизотропии для широкой группы горных пород. Кросс-дипольные измерения также могут быть использованы для определения ориентации крупных трещин и обнаружения анизотропии, индуцированной подземными горизонтальными напряжениями и трещиноватостью \cite{Patterson2001}.

Как известно, в процессе измерений в трансверсально-изотропной (ТИ) породе наблюдается поляризация распространяющихся по стволу скважины поперечных волн. При наличии измерений от двух направленных ортогонально-ориентированных источников в скважине возможно определить направления главных осей ТИ модели и скорости распространения поперечных волн. В основе классического метода определения лежит допущение о симметричности матрицы измерений (составленной из данных четырёх измерений с различной ориентацией источников и приёмников), которая может быть приведена к диагональному виду ортогональным преобразованием \cite{Alford1986}. 

Однако, зачастую ортогональность направлений поляризации поперечных волн отсутствует. Во многих случаях, например при распространении волн в анизотропной породе с орторомбическим типом симметрии \cite{Dellinger2001} или в случае анизотропии вызванной наличием трещин \cite{Nolte1996}, известно, что поляризация волн может быть существенно неортогональной.
%Такой эффект может наблюдаться в породах с орторомбическим типом симметрии, а также при существовании нескольких дополнительных факторов образования выделенных направлений, таких как трещины или сильные горизонтальные напряжения \cite{Nolte1997}. 
Чтобы учесть возможную неортогональность поперечных волн в однородной породе в \cite{Dellinger1998} был рассмотрен способ диагонализации матрицы измерений некоторым неортогональным преобразованием, обобщающий \cite{Alford1986} на этот случай. %Корректное применение этой процедуры позволяет оценить неортогональные главные направления анизотропной породы, вызванные, например, трещиноватостью. 

%В то же время, фактором, влияющим на поляризацию и разделение волн в скважине, является нецилиндрическая форма поперечного сечения ствола \cite{Seroices2010}. Поэтому причиной неортогональности может стать простое сочетание ТИ породы и фактора нецилиндричности. Необходимо различать неортогональность направлений поляризации поперечных волн вследствие указанного сочетания двух факторов и вследствие нарушения ТИ модели.
В скважинах с деформированным стволом, в частности эллиптического сечения, дипольные моды распространяются с разными скоростями \cite{Seroices2010}. Для таких скважин в анизотропной породе предположения классического метода, а также его неортогонального обобщения, о независимом распространении волн также не выполняются (кроме случая, когда главные направления анизотропной породы совпадают с осями эллипса). Сильное нарушение симметрии задачи не позволяет выделить направления поляризации мод и определить главные направления анизотропной породы. 

В данной работе с помощью численного моделирования исcледуется вопрос оценки погрешности определения главных направлений ТИ породы по измерениям в скважинах нецилиндрического сечения. Целями исследования являются 1) выявление признаков, свидетельствующих о наличии большой ошибки в результатах определения главных направлений анизотропии при обработке методом Alford rotation%(т.е. когда выявляется неортогональность для исходной ортогональности)
, 2) изучение возможностей частотной фильтрации и "неортогональных" алгоритмов обработки для получения наиболее достоверных результатов по определению главных направлений, 3) разработка методики применения полуаналитического метода конечных элементов (SAFE) \cite{Bartoli2006} для моделирования рассматриваемых задач и анализа структуры волнового поля в скважинах. %В качестве исходных данных используются результаты численного трёхмерного моделирования с помощью метода спектральных элементов \cite{Komatitsch2000}. Результаты их обработки классическим алгоритмом Alford rotation \cite{Alford1986} и его неортогональной модификацией \cite{Dellinger1998} сопоставляются с расчётами поляризации волн методом SAFE. % применением частотной фильтрации и без неё.

\section{Ортогональный и неортогональный алгоритмы Alford rotation}
В устройство классического прибора акустического дипольного каротажа входят два источника и массива приёмников направленного действия, ортогонально ориентированных друг к другу. Свяжем с ориентацией источников и приёмников оси X и Y локальной системы координат прибора. В ходе работы прибора на выходе получают четыре разных массива значений акустического сигнала от времени, обозначаемых XX, XY, YX и YY, где первая буква обозначает активный в момент проведения источник, а вторая - массив приёмников. Данные, полученные с приёмников на определённом расстоянии от источника, принято записывать в форме матрицы $\mathbf{R}$ 

\begin{equation}
	\mathbf{R} = \left\|
	\begin{array}{cc}
	\text{XX} & \text{YX} \\
	\text{XY} & \text{YY} \\
	\end{array}
	\right\| 
	\label{eq:R_matrix}
\end{equation}

Известно, что в однородной недисперсионной среде с ТИ типом симметрии в произвольном направлении могут распространятся три вида плоских волн (квазипродольная, поперечная и квазипоперечная) с ортогональными векторами поляризации%\cite{Musgrave1970}
. Дипольные излучатели в скважине возбуждают преимущественно моды с поперечным характером колебаний, которые распространяются вдоль скважины независимо друг от друга. В рамках классического подхода для описания их распространения используется модель плоских волн.  В данных предположениях исходную матрицу измерений \eqref{eq:R_matrix} возможно приближённо представить в виде \cite{Dellinger1998}
\begin{equation}
	\mathbf{R} \approx \mathbf{P} \ \mathbf{D} \ \mathbf{P}^T, \label{eq:alford_symmetric} 
\end{equation}
где $\mathbf{D}$ - диагональная матрица, содержащая чистые сигнатуры двух дипольных мод; матрица $\mathbf{P}$ проецирует сигналы отдельных мод на оси локальной системы координат прибора. Так как поляризации плоских волн ортогональны, то матрица преобразования $\mathbf{P}$ сводится к повороту на некоторый угол $\theta$

\begin{equation*}
	\mathbf{P} = \left\|
	\begin{array}{cc}
	\cos \theta &-\sin \theta \\ 
	\sin \theta & \cos \theta
	\end{array} 
	\right\| 
\end{equation*}
Алгоритм поиска этого угла был представлен в работе \cite{Alford1986} и получил название Alford rotation.

Для случаев, когда поляризация изгибных мод не является ортогональной, один из возможных вариантов обобщения Alford rotation был предложен в работе \cite{Dellinger1998} и заключается в введении дополнительного угла $\eta$, характеризующего ориентацию главных направлений. Для приближенного представления \eqref{eq:alford_symmetric} матрица преобразования будет иметь вид:
\begin{align*}
\mathbf{P} &= \left\|
\begin{array}{cc}
\cos \theta & -\sin (\theta+\eta) \\ 
\sin \theta & \cos (\theta+\eta)
\end{array} 
\right\|
\end{align*}
где за $\theta$ принимается угол? отсчитываемый против часовой стрелки между осью $X$ и направлением поляризации первой моды, а за $\theta + \eta$ - угол между направлением поляризации второй моды и осью $Y$. При $\eta=0$ метод сводится к классическому Alford rotation. При обработке данных моделирования в данной работе используются следующие обозначения для найденных углов: ортогональный Alford rotation: $\theta_1^o=\theta$, $\theta_2^o=\theta-90$; неортогональный Alford rotation: $\theta_1^n=\theta$, $\theta_2^n=\theta+\eta$. Поиск значений углов производится через минимизацию энергии недиагональных компонент матрицы $\mathbf{D}$ по двум параметрам.

%низкочастотные и высокочастотные фильтры с конечной импульсной характеристикой. 
%В разделе \ref{comparison_alford} производится оценка точности ортогонального и неортогонального Alford rotation на основе синтетических данных трехмерного моделирования распространения волн в быстрой ТИ породы в скважинах эллиптического сечения.  

\section{Вычислительные методы}
В качестве исходных данных каротажных измерений используются результаты прямого моделирования распространения волн методом спектральных элементов (SEM). Ранее данный метод успешно применялся для расчёта задач геофизики \cite{Komatitsch1999} и моделирования акустического каротажа \cite{Charara2011}. Численный алгоритм производит решение нестационарных уравнений колебаний линейно-упругой анизотропной породы и уравнений акустики для невязкой жидкости внутри скважины с соответствующими условиями на границе раздела фаз. Подробное описание и формулировка метода приводится в работе \cite{Komatitsch1999}.

Основная обработка и анализ данных, в том числе и Alford rotation, проводились средствами MATLAB. Для построения дисперсионных кривых нормальных мод использовался алгоритм, основанный на модифицированном методе Прони \cite{Ekstrom1995}. При обработке данных измерений в некоторых случаях применялись оконные, низкочастотные и высокочастотные фильтры сигнала, реализованные в MATLAB. 

Для анализа решения в частотной области в предположении однородности среды и геометрии по оси Z был выбран более простой и быстрый по сравнению с SEM полуаналитический метод конечных элементов (SAFE) \cite{Bartoli2006}. Формулировка метода основана на Фурье разложения искомой функции вдоль направления оси скважины, что позволяет свести задачу к набору двухмерных постановок. Приведем краткое описание метода. Предполагая гармоническую зависимость от времени вида $\mathrm{e}^{-\ii\omega t}$ для смещений $\mathbf{u}$, деформаций $\boldsymbol{\varepsilon}$ и напряжений $\boldsymbol{\sigma}$, уравнения движения твёрдого тела в вариационной форме могут быть представлены в виде:

\begin{equation}
\int_{V}^{(s)}\delta \boldsymbol{\varepsilon}^* \boldsymbol{\sigma} dV - \omega^2 \int_{V}^{(s)} \rho_s \delta \mathbf{u}^*\mathbf{u}dV = \int_{V}^{(s)}\delta \mathbf{u}^* \mathbf{f} dV + \int_{\partial V}^{(s)}\delta \mathbf{u}^* \mathbf{t} d\Gamma, \label{var_eq_solid}
\end{equation}
здесь $\mathbf{f}$, $\mathbf{t}$ -- векторы объёмных и поверхностных сил, $\rho_s$ -- плотность, тензор напряжений $\boldsymbol{\sigma}$ связан с тензором деформаций $\boldsymbol{\varepsilon}$ для упругого тела через закон Гука:
$$
\boldsymbol{\sigma} = \mathbf{C}\boldsymbol{\varepsilon}.
$$

При описании движения невязкой жидкости будем пользоваться формулировкой уравнений в терминах потенциала скорости $\phi$: $\ii \omega \mathbf{u}_f = \nabla \phi$. Тогда давление в жидкости определяется выражением $p = -\ii \omega \rho_f \phi$, а уравнения движения для жидкой среды в вариационной форме имеют вид: 

\begin{equation}
\int_{V}^{(f)} \delta (\nabla\phi)^* \rho_f  \nabla \phi dV - \omega^2 \int_{V}^{(f)}  c^{-2} \rho_f \delta \phi^*  \phi dV = \frac{1}{\ii\omega}\int_{\partial V}^{(f)} \rho_f \delta(\nabla \phi)^* \mathbf{t} d\Gamma + \frac{1}{\ii\omega} \int_{V}^{(f)} \delta(\nabla \phi)^* \mathbf{f} dV, \label{var_eq_fluid}
\end{equation}
где $c$ -- скорость звука в жидкости.

Свяжем вертикальную ось скважины с направлением оси Z системы координат и применим преобразование Фурье по $z$ к исходным уравнениям. Для каждого элемента из сетки конечных элементов в плоскости поперечного сечения скважины значения искомых величин аппроксимируем системой базисных функций $N_j(x,y)$ \cite{Zienkiewicz2000}:

\begin{equation}
\begin{split}
\mathbf{u}^{(e)}(x,y,z,t) &= \left[
\begin{array}{c}
\sum_{j=1}^{n}N_j(x,y)U_{x}^{(j)} \\
\sum_{j=1}^{n}N_j(x,y)U_{y}^{(j)} \\
\sum_{j=1}^{n}N_j(x,y)U_{z}^{(j)} 
\end{array}
\right] \mathrm{e}^{\ii (kz-\omega t)} % = \mathbf{N}_u(x,y) \mathbf{U}^{(e)}(k,\omega) e^{\ii (kz-\omega t)}, 
\\
\phi^{(e)}(x,y,z,t) &= \left(\sum_{j=1}^{n}N_j(x,y)\psi^{(j)} \right) \mathrm{e}^{\ii (kz-\omega t)} % = \mathbf{N}_{\phi}(x,y) \mathbf{\Phi}^{(e)}(k,\omega) e^{\ii (kz-\omega t)}, 
\\
\end{split} \label{eq:unknown_var}
\end{equation}
где $n$ -- число узлов в элементе c номером $e$.  

С учётом условий на границе раздела жидкости и твёрдого тела при подстановке неизвестных \eqref{eq:unknown_var} в уравнения \eqref{var_eq_solid} и \eqref{var_eq_fluid} задача сводится к системе линейных уравнений \cite{Bartoli2006,Treyssede2013}:
\begin{equation}
(\mathbf{K}_1 + \ii k \mathbf{K}_2 + k^2 \mathbf{K}_3 - \omega^2 \mathbf{M} + \ii \omega \mathbf{P}) \hat{\mathbf{U}} = \hat{\mathbf{F}} \label{eq:eigen_equation}
\end{equation}
где матрицы $\mathbf{K}_1$, $\mathbf{K}_2$, $\mathbf{K}_3$, $\mathbf{M}$, $\mathbf{P}$ формируются из значений объёмных и поверхностных интегралов в уравнениях \eqref{var_eq_solid} и \eqref{var_eq_fluid} на элементах, а $ \hat{\mathbf{U}}$ состоит из значений искомых величин $\mathbf{U}^{(j)}$ и $\psi^{(j)}$ в узлах каждого элемента. 

Для каждого заданного значения частоты $\omega$ формулируется обобщённая задача на собственные значения для матрицы уравнения \eqref{eq:eigen_equation}, решением которой являются пары собственных значений и векторов $[k_m, \hat{\mathbf{U}}_m]$, соответствующие различным волновым модам системы. Специальный отбор собственных векторов позволяет выделить компоненты волнового поля, соответствующие дипольным модам в скважине. Градиент значений выбранного собственного вектора внутри скважины указывает направление колебаний частиц для конкретной моды. Если на некотором диапазоне частот эти направления меняются незначительно, будем говорить о поляризации такой локализованной в частотной области волны. Сравнение таких модельных направлений поляризации с результатами, полученными прямой обработкой временных сигналов с приёмников, рассчитанных методом спектральных элементов, приведены в разделе \ref{safe_comparison}.

\section{Результаты обработки Alford rotation}
\label{comparison_alford}
Для имитации результатов каротажных измерений использовалась трёхмерная численная модель распространения волн. В качестве сигнала по времени дипольного акустического источника в скважине была взята производная вейвлета Блэкмана-Харриса с центральной частотой 4 кГц. Матрица измерений $\mathbf{R}$ вообще говоря не является строго симметричной из-за наличия шума в сигналах. Чем меньше отношение $E_{rel}$ суммы квадратов значений недиагональных компонент, полученной после преобразования матрицы $\mathbf{D}$ к диагональным, тем лучше работает алгоритм диагонализации и тем достовернее значения полученных углов. При обработке эталонных имитируемых измерений в цилиндрической скважине величина $E_{rel}$ имеет значения порядка $10^{-7}$; угол $\theta$ совпадает с заданным модельным значением до десятых долей градуса. Эти значения имеют порядок погрешности численного метода и алгоритма обработки. 

Для оценки влияния несимметричности формы скважины на результат работы Alford rotation и его неортогонального обобщения были рассмотрены примеры эллиптических скважин в ТИ породах  Bakken Shale и Cotton Valey Shale. Обе породы относятся к классу глинистых сланцев и имеют скорость распространения поперечных волн, превышающую скорость звука в жидкости в скважине (т. н. быстрые породы). Значение упругих постоянных материалов приведены в таблице \ref{tab:properties}. Ось симметрии ТИ породы наклонена по отношению к оси скважины под углом 90\textdegree (горизонтальная трансверсальная изотропия), искомый угол поворота оси симметрии в плоскости поперечного сечения скважины равен $\theta = 45$\textdegree. В расчётах использовались скважины c размерами полуосей $12.70 \times 10.16$ см ($5 \times 4$ дюймов) и $15 \times 10$ см. 
%Полное описание рассматриваемых моделей можно найти в таблице \ref{tab:models_description}. 

%Параметры быстрой породы, $\rho = 2640$ кг/м$^3$, $V_P = 5219$ м/с, $V_S = 3257$ м/с.
\begin{table}[H]
\footnotesize
%\centering
\caption{Параметры упругих анизотропных материалов}
\renewcommand{\arraystretch}{1.5}
\begin{tabularx}{\textwidth}{|C|c|c|c|c|c|c|c|}
\hline \multirow{2}{*}{Название}  & Плотность & \multicolumn{6}{c|}{Модули упругости, ГПа} \\ 
\cline{3-8}  & кг/м$^3$ & $C_{11}$ & $C_{12}$ & $C_{13}$ & $C_{33}$ & $C_{44}$ & $C_{66}$ \\ \hline
\hline Cotton Valey Shale & 2640 & 74.73 & 14.75 & 25.29 & 58.84 & 22.05 & 29.99 \\ 
\hline Bakken Shale & 2230 & 40.9 & 10.3 & 8.5 & 26.9 & 10.5 & 15.3 \\ 
\hline 
\end{tabularx} 
\label{tab:properties}
\renewcommand{\arraystretch}{1.0}
\end{table}

Как можно видеть, оба алгоритма на исходных модельных данных дают примерно одинаковые результаты (Таблица \ref{tab:std_process_results}). Полученные главные направления почти ортогональны, но значительно отличаются от заданных в модели значений 45\textdegree \ и -45\textdegree. В недиагональных компонентах при этом остаётся достаточно мало - от 1\% до 3\% - энергии. Таким образом, Alford rotation в скважинах с деформированным стволом может давать направления, не совпадающие с физическими.

Ранее было показано \cite{Seroices2010}, что в эллиптических скважинах на низких частотах форма скважины почти не оказывает влияние на распространение волн, а определяется свойствами породы. Рассмотрим вопрос, как результат обработки зависит от частотного спектра приходящего сигнала. Как известно, характерной особенностью распространения дипольных колебаний в скважине является большая дисперсия. Используя модифицированный метод Прони \cite{Ekstrom1995}, были построены дисперсионные кривые для гармоник исходного сигнала с наиболее высокой амплитудой. В моделируемых задачах они соответствуют двум главным дипольным модам. %(рис. \ref{fig:disp_curves_all}). Аналогичные кривые также были построены по результатам расчётов полуаналитическим методом конечных элементов (SAFE) и нанесены на графики для проверки точности. 
По этим кривым был выделен диапазон низких частот, где скорость распространения волны стационарна и близка к одной из скоростей поперечных волн в породе, а также диапазон высоких частот с относительно постоянной скоростью распространения. 

Как видно из таблицы \ref{tab:filter_process_results}, полученные оценки углов гораздо ближе к модельным значениям, чем значения от нефильтрованного сигнала. Также отметим, что при обработке фильтрованных данных наблюдается заметная неортогональность между направлениями поляризации дипольных мод на низких частотах. Применение высокочастотных фильтров, напротив приводит почти полностью ортогональному результату, близкому к направлениям осей эллипса. В следующем разделе более подробно исследуется вопрос связи приведённых результатов с физической поляризацией распространяющихся волн.
%Более подробное рассмотрение характера колебаний в переходной средней частотной зоне выходит за рамки данной работы. 

\begin{table}[h]
\footnotesize
%\centering
\caption{Результаты обработки исходных данных алгоритмами Alford rotation}
\renewcommand{\arraystretch}{1.5}
\begin{tabularx}{\textwidth}{|X|rr|rr|r|rr|}
\hline	
				&\multicolumn{1}{c}{$\theta_1^o$} & \multicolumn{1}{c|}{$\theta_1^n$} & \multicolumn{1}{c}{$\theta_2^o$} & \multicolumn{1}{c|}{$\theta_2^n$} & \multicolumn{1}{c|}{$\Delta\theta^n$}& \multicolumn{1}{c}{$E_{rel}^o, \%$} & \multicolumn{1}{c|}{$E_{rel}^n, \%$} \\ \hline
\hline Bakken Shale ($12.70 \times 10.16$) & 15.6 & 14.7 & -74.4 & -73.9  & 1.4  & 3.0 & 3.0 \\
\hline Bakken Shale ($15.00 \times 10.00$) & 8.4 & 8.1 & -81.6 & -81.6 & 0.3 & 1.7 & 1.7 \\
\hline Cotton Valey Shale ($12.70 \times 10.16$) & 3.3 & 3.0 & -86.7 & -86.7  & 0.3 & 0.8 & 0.8 \\ 
\hline Cotton Valey Shale ($15.00 \times 10.00$) & 1.6 & 1.8 & -88.4 & -88.4  & 0.0  & 0.6 & 0.6 \\	   
 	   \hline
\end{tabularx} 
\begin{flushleft}
* здесь $\theta_1^o,\theta_2^o$ и $\theta_1^n,\theta_2^n$ соответствуют результатам, полученным ортогональной и неортогональной версией алгоритма. Величина $E_{rel}$ обозначает отношение энергии недиагональных компонент матрицы измерений к полной энергии.
\end{flushleft}
\label{tab:std_process_results}
\renewcommand{\arraystretch}{1.0}
\end{table}

\begin{table}[h]
\footnotesize
\centering
\caption{Результаты расчетов с применением фильтров}
\renewcommand{\arraystretch}{1.5}
\begin{tabularx}{\textwidth}{|X|rr|rr|r|rr|}
\hline
				&\multicolumn{1}{c}{$\theta_1^o$} & \multicolumn{1}{c|}{$\theta_1^n$} & \multicolumn{1}{c}{$\theta_2^o$} & \multicolumn{1}{c|}{$\theta_2^n$} & \multicolumn{1}{c|}{$\Delta\theta^n$}& \multicolumn{1}{c}{$E_{rel}^o, \%$} & \multicolumn{1}{c|}{$E_{rel}^n, \%$} \\ \hline
\hline	\textbf{Bakken Shale ($12.70 \times 10.16$)} & \textbf{15.6} & \textbf{14.7} & \textbf{-74.4}  & \textbf{-73.9}  & \textbf{1.4}  & \textbf{3.0} & \textbf{3.0} \\
		-//-//- с НЧФ & 49.4 & 40.5 & -40.6 & -35.4  & 14.1 & 2.0 & 1.2\\
		-//-//- с ВЧФ & 14.0 & 13.3 & -76.0 & -75.6  & 1.2 & 0.4 & 0.4\\
\hline	\textbf{Bakken Shale ($15.00 \times 10.00$)} & \textbf{8.4} & \textbf{8.1} & \textbf{-81.6}  & \textbf{-81.6} & \textbf{0.3}  & \textbf{1.7} & \textbf{1.7} \\
		-//-//- с НЧФ & 41.2 & 25.7 & -48.8 & -32.0  & 32.3 & 22.1 & 12.0\\
		-//-//- с ВЧФ & 7.5 & 10.2 & -82.5 & -82.6  & 2.8 & 1.1 & 1.0\\
\hline	\textbf{Cotton Valey Shale ($12.70 \times 10.16$)} & \textbf{3.3} & \textbf{3.0} & \textbf{-86.7}  & \textbf{-86.7}  & \textbf{0.3}  & \textbf{0.8} & \textbf{0.8}\\
		-//-//- с НЧФ & 48.4 & 39.7 & -41.6 & -35.5  & 14.8  & 9.7 & 7.0 \\
		-//-//- с ВЧФ & 2.8 & 3.3 & -87.2 & -87.4  & 0.7  & 0.5 & 0.5\\	
\hline	\textbf{Cotton Valey Shale ($15.00 \times 10.00$)} & \textbf{1.6} & \textbf{1.8} & \textbf{-88.4}  & \textbf{-88.4}  & \textbf{0.00}  & \textbf{0.6} & \textbf{0.6} \\
		-//-//- с НЧФ & 6.0 & 7.7 & -84.0 & -56.6  & 25.7  & 7.9 & 7.3 \\
		-//-//- с ВЧФ & 1.5 & 2.0 & -88.5 & -88.7  & 0.7  & 1.8 & 1.8\\		
\hline	
\end{tabularx} 
\begin{flushleft}
* здесь $\theta_1^o,\theta_2^o$ и $\theta_1^n,\theta_2^n$ соответствуют результатам, полученным ортогональной и неортогональной версией алгоритма. Величина $E_{rel}$ обозначает отношение энергии недиагональных компонент матрицы измерений к полной энергии.
\end{flushleft}
\label{tab:filter_process_results}
\renewcommand{\arraystretch}{1.0}
\end{table}


\section{Сравнение решений SAFE с результатами обработки решений SEM}
\label{safe_comparison}

Полуаналитический метод конечных элементов использует представление решения для волнового поля в скважине волн в виде суммы мод. Основная энергия в волне, возбуждаемой дипольным источником, содержится в дипольных модах, поэтому из всего набора собственных векторов были выбраны те, которые соответствуют именно этим колебаниям. Градиент значений собственного вектора в скважине позволяет получить направление преимущественного колебания частиц на некоторой частоте. В породах, выбранных для моделирования, эти направления существенно меняются в районе от 4 до 6 кГц \cite{Zharnikov2015}. Но в зонах пропускания используемых низкочастотных и высокочастотных фильтров изменения направлений градиентов собственных векторов невелики. Для сравнения с результатами обработки Alford rotation (Таблица \ref{tab:filter_process_results}) были выбраны собственные вектора для дипольных мод на частоте, соответствующей максимумам энергии в спектре фильтрованных данных (см. Рис~\ref{fig:comparison_safe_all}).

\begin{figure}[h]
\centering
%\begin{minipage}{0.49\linewidth}
%	\centering \textbf{Bakken Shale ($12.70 \times 10.16$)}
%	\psfragfig[width=0.49\linewidth,crop=pdfcrop]{./images/nonorth_alford/el10x8_inch_HTI_BS_f45_disp_modes+SAFE} \\
%\end{minipage}
\begin{minipage}{0.49\linewidth}
	\centering \textbf{Bakken Shale} \\ ($15.00 \times 10.00$)
	\psfragfig[width=0.49\linewidth,crop=pdfcrop]{./images/nonorth_alford/el15x10_HTI_BS_f45_disp_modes+SAFE_gr} \\
\end{minipage}
%\begin{minipage}{0.49\linewidth}
%	\centering \textbf{Cotton Valey Shale ($12.70 \times 10.16$)}
%	\psfragfig[width=0.49\linewidth,crop=pdfcrop]{./images/nonorth_alford/el10x8_inch_HTI_CS_f45_disp_modes+SAFE} \\
%\end{minipage}
\begin{minipage}{0.49\linewidth}
	\centering \textbf{Cotton Valey Shale} \\ ($15.00 \times 10.00$)
	\psfragfig[width=0.49\linewidth,crop=pdfcrop]{./images/nonorth_alford/el15x10_HTI_CS_f45_disp_modes+SAFE_new_gr} \\
\end{minipage}
\caption{Дисперсионные кривые для основных задач. }
\label{fig:disp_curves_all}
\end{figure}

\begin{figure}[h]
\centering
\renewcommand{\arraystretch}{1.5}
\begin{tabular*}{1\textwidth}{c|cc|cc|}
\cline{2-5}
&\multicolumn{2}{c|}{\textbf{Bakken Shale ($12.70 \times 10.16$)}} &\multicolumn{2}{c|}{\textbf{Bakken Shale ($15.00 \times 10.00$)}}\\ 
\begin{minipage}{0.02\textwidth}
\rotatebox{90}{\footnotesize \textit{Дипольная мода 1}} 
\end{minipage}&
\begin{minipage}{0.22\textwidth}
	\psfragfig[width=0.23\textwidth,crop=pdfcrop]{./images/SAFE/SAFE_BS_10x8_HTI_45/P_s_3_3kHz_gr}		
\end{minipage}&
\begin{minipage}{0.22\textwidth}
	\psfragfig[width=0.23\textwidth,crop=pdfcrop]{./images/SAFE/SAFE_BS_10x8_HTI_45/P_s_5_5kHz_gr}		
\end{minipage}&
\begin{minipage}{0.22\textwidth}
	\psfragfig[width=0.23\textwidth,crop=pdfcrop]{./images/SAFE/SAFE_BS_15x10_HTI_45/P_s_3_3kHz_gr}		
\end{minipage}&
\begin{minipage}{0.22\textwidth}
	\psfragfig[width=0.23\textwidth,crop=pdfcrop]{./images/SAFE/SAFE_BS_15x10_HTI_45/P_s_5_5kHz_gr}		
\end{minipage}\\ 
%& & & & \\
\begin{minipage}{0.02\textwidth}
\rotatebox{90}{\footnotesize \textit{Дипольная мода 2}} 
\end{minipage} &
\begin{minipage}{0.22\textwidth}
	\psfragfig[width=0.23\textwidth,crop=pdfcrop]{./images/SAFE/SAFE_BS_10x8_HTI_45/P_a_3_3kHz_gr}		
\end{minipage} &
\begin{minipage}{0.22\textwidth}
	\psfragfig[width=0.23\textwidth,crop=pdfcrop]{./images/SAFE/SAFE_BS_10x8_HTI_45/P_a_5_5kHz_gr}		
\end{minipage} &
\begin{minipage}{0.22\textwidth}
	\psfragfig[width=0.23\textwidth,crop=pdfcrop]{./images/SAFE/SAFE_BS_15x10_HTI_45/P_a_3_3kHz_gr}		
\end{minipage} &
\begin{minipage}{0.22\textwidth}
	\psfragfig[width=0.23\textwidth,crop=pdfcrop]{./images/SAFE/SAFE_BS_15x10_HTI_45/P_a_5_5kHz_gr}		
\end{minipage} \\ 
& \footnotesize НЧФ, 3.29 кГц & \footnotesize ВЧФ, 5.53 кГц & \footnotesize НЧФ, 3.29 кГц & \footnotesize ВЧФ, 5.53 кГц \\ \cline{2-5}
\end{tabular*} \\
{А)} \\
\quad \\
\begin{tabular*}{1\textwidth}{c|cc|cc|}
\cline{2-5}
&\multicolumn{2}{c|}{\textbf{Cotton Valey Shale ($12.70 \times 10.16$)}} &\multicolumn{2}{c|}{\textbf{Cotton Valey Shale ($15.00 \times 10.00$)}}\\
\begin{minipage}{0.02\linewidth}
	\rotatebox{90}{\footnotesize\textit{Дипольная мода 1}} 
\end{minipage}&
\begin{minipage}{0.22\linewidth}
	\psfragfig[width=0.22\linewidth,crop=pdfcrop]{./images/SAFE/SAFE_CS_10x8_HTI_45/P_s_3_0kHz_gr}		
\end{minipage}&
\begin{minipage}{0.22\linewidth}
	\psfragfig[width=0.22\linewidth,crop=pdfcrop]{./images/SAFE/SAFE_CS_10x8_HTI_45/P_s_7_2kHz_gr}		
\end{minipage}&
\begin{minipage}{0.22\linewidth}
	\psfragfig[width=0.22\linewidth,crop=pdfcrop]{./images/SAFE/SAFE_CS_15x10_HTI_45/P_s_3_0kHz_gr}		
\end{minipage}&
\begin{minipage}{0.22\linewidth}
	\psfragfig[width=0.22\linewidth,crop=pdfcrop]{./images/SAFE/SAFE_CS_15x10_HTI_45/P_s_7_2kHz_gr}		
\end{minipage} \\
\begin{minipage}{0.02\linewidth}
	\rotatebox{90}{\footnotesize\textit{Дипольная мода 2}} 
\end{minipage}&
\begin{minipage}{0.22\linewidth}
	\psfragfig[width=0.22\linewidth,crop=pdfcrop]{./images/SAFE/SAFE_CS_10x8_HTI_45/P_a_3_0kHz_gr}		
\end{minipage}&
\begin{minipage}{0.22\linewidth}
	\psfragfig[width=0.22\linewidth,crop=pdfcrop]{./images/SAFE/SAFE_CS_10x8_HTI_45/P_a_7_2kHz_gr}		
\end{minipage}&
\begin{minipage}{0.22\linewidth}
	\psfragfig[width=0.22\linewidth,crop=pdfcrop]{./images/SAFE/SAFE_CS_15x10_HTI_45/P_a_3_0kHz_gr}		
\end{minipage}&
\begin{minipage}{0.22\linewidth}
	\psfragfig[width=0.22\linewidth,crop=pdfcrop]{./images/SAFE/SAFE_CS_15x10_HTI_45/P_a_7_2kHz_gr}		
\end{minipage}\\
& \footnotesize НЧФ, 3.03 кГц & \footnotesize ВЧФ, 7.17 кГц & \footnotesize НЧФ, 3.03 кГц & \footnotesize ВЧФ, 7.17 кГц \\ \cline{2-5}
\end{tabular*}
\\
{Б)} \\
\quad \\
\renewcommand{\arraystretch}{1.0}
\footnotesize
\begin{tabular*}{\textwidth}{@{\extracolsep{\fill} }crccc}
& 						 	& \tikz \draw (0,0) -- (1cm,0);  	& \tikz \draw[dashed] (0,0) -- (1cm,0);  	& \tikz \draw[very thick,dashdotted] (0,0) -- (1cm,0); \\
& Результаты обработки 		& \textit{ортогональный} 			& \textit{ортогональный} 					& \textit{неортогональный}    			\\
& Alford rotation:			& \textit{без фильтрации}		 	& \textit{с фильтрацией} 					& \textit{с фильтрацией} 	\\
\end{tabular*}
\renewcommand{\arraystretch}{1.0}
\normalsize
\caption{Сравнение результатов обработки фильтрованных данных измерений и значений собственных векторов дипольных мод на частоте, соответствующей максимуму энергии в спектре сигнала, в породе Bakken Shale (А) и Cotton Valey Shale (Б). Цветом показано нормированное на интервале [0,1] значение собственного вектора (давление) для указанной частоты. НЧФ и ВЧФ обозначают тип фильтрации - низкочастотная и высокочастотная - применённой к исходным данным.}
\label{fig:comparison_safe_all}
\end{figure}

%\begin{figure}[h]
%%\centering
%\renewcommand{\arraystretch}{1.5}
%\begin{tabular*}{1\textwidth}{c|cc|cc|}
%\cline{2-5}
%&\multicolumn{2}{c|}{\textbf{Cotton Valey Shale ($12.70 \times 10.16$)}} &\multicolumn{2}{c|}{\textbf{Cotton Valey Shale ($15.00 \times 10.00$)}}\\
%\begin{minipage}{0.02\linewidth}
%\rotatebox{90}{\footnotesize\textit{Дипольная мода 1}} 
%\end{minipage}&
%\begin{minipage}{0.22\linewidth}
%	\psfragfig[width=0.22\linewidth,crop=pdfcrop]{./images/SAFE/SAFE_CS_10x8_HTI_45/P_s_3_0kHz}		
%\end{minipage}&
%\begin{minipage}{0.22\linewidth}
%	\psfragfig[width=0.22\linewidth,crop=pdfcrop]{./images/SAFE/SAFE_CS_10x8_HTI_45/P_s_7_2kHz}		
%\end{minipage}&
%\begin{minipage}{0.22\linewidth}
%	\psfragfig[width=0.22\linewidth,crop=pdfcrop]{./images/SAFE/SAFE_CS_15x10_HTI_45/P_s_3_3kHz}		
%\end{minipage}&
%\begin{minipage}{0.22\linewidth}
%	\psfragfig[width=0.22\linewidth,crop=pdfcrop]{./images/SAFE/SAFE_CS_15x10_HTI_45/P_s_5_5kHz}		
%\end{minipage} \\
%\begin{minipage}{0.02\linewidth}
%\rotatebox{90}{\footnotesize\textit{Дипольная мода 2}} 
%\end{minipage}&
%\begin{minipage}{0.22\linewidth}
%	\psfragfig[width=0.22\linewidth,crop=pdfcrop]{./images/SAFE/SAFE_CS_10x8_HTI_45/P_a_3_0kHz}		
%\end{minipage}&
%\begin{minipage}{0.22\linewidth}
%	\psfragfig[width=0.22\linewidth,crop=pdfcrop]{./images/SAFE/SAFE_CS_10x8_HTI_45/P_a_7_2kHz}		
%\end{minipage}&
%\begin{minipage}{0.22\linewidth}
%	\psfragfig[width=0.22\linewidth,crop=pdfcrop]{./images/SAFE/SAFE_CS_15x10_HTI_45/P_a_3_3kHz}		
%\end{minipage}&
%\begin{minipage}{0.22\linewidth}
%	\psfragfig[width=0.22\linewidth,crop=pdfcrop]{./images/SAFE/SAFE_CS_15x10_HTI_45/P_a_5_5kHz}		
%\end{minipage}\\
%& \footnotesize НЧФ, 3.03 кГц & \footnotesize ВЧФ, 7.17 кГц & \footnotesize НЧФ, 3.03 кГц & \footnotesize ВЧФ, 7.17 кГц \\ \cline{2-5}
%\end{tabular*}
%\renewcommand{\arraystretch}{1.0}
%\footnotesize
%\begin{tabular*}{\textwidth}{@{\extracolsep{\fill} }crccc}
%& 						 	& \tikz \draw (0,0) -- (1cm,0);  	& \tikz \draw[dashed] (0,0) -- (1cm,0);  	& \tikz \draw[dashdotted] (0,0) -- (1cm,0); \\
%& Результаты обработки 		& \textit{ортогональный} 			& \textit{ортогональный} 					& \textit{неортогональный}    			\\
%& Alford rotation:			& \textit{без фильтрации}		 	& \textit{с фильтрацией} 					& \textit{с фильтрацией} 	\\
%\end{tabular*}
%\renewcommand{\arraystretch}{1.0}
%\normalsize
%\caption{Сравнение результатов обработки фильтрованных данных измерений в породе Cotton Valey Shale и направлений поляризации двух дипольных мод, полученными SAFE. Здесь НЧФ и ВЧФ обозначают тип примененной к исходным данным фильтрации (низкочастотная и высокочастотная соответственно), значение частоты соответствует отображенному цветом решению SAFE для собственного вектора дипольной моды.}
%\label{fig:cs15_10_HTI45}
%\end{figure}

Представленные данные хорошо демонстрируют неортогональность рассчитанных направлений колебаний дипольных мод на низких частотах. Результаты, полученные неортогональной версией Alford rotation, хорошо согласуются как качественно, так и количественно с направлениями градиентов собственных векторов. Заметим также, что в этом частотном диапазоне результат практически не отличается для двух выбранных геометрий скважин. При этом полученные направления не совпадают с заданной в изначальной модели ориентацией оси симметрии трансверсально-изотропной породы. Этот факт подтверждает неприменимость классического подхода оценки главных направлений ТИ породы в задачах такого типа. 

Отметим, что близкие к 45\textdegree \ значения угла классического ортогонального Alford rotation являются лишь случайным совпадением осредненных реальных поляризаций мод на этих частотах с заданным значением в модели. В пользу этого утверждения говорит факт, что энергия недиагональных компонент при ортогональной обработке достаточно велика - до 10\% (см. Таблицу \ref{tab:filter_process_results}). 

Результаты обработки нефильтрованного сигнала в рассмотренных задачах, как видно из данных таблиц, в целом дают близкие значения результатам обработки сигнала после применения высокочастотного фильтра. Интересно, что при этом поляризация мод почти ортогональна, но не совпадает с направлениями полуосей эллипса поперечного сечения скважины. При увеличении степени эллиптичности ствола это различие уменьшается. Таким образом, даже при корректной (с точки зрения диагонализации матрицы измерений) работе алгоритма полученное значение угла на направление главной оси анизотропного материала может не отвечать ни физическим свойствам породы, ни геометрической ориентации скважины. 


\section{Заключение}
Приведённые в работе расчёты показывают, что деформация ствола скважины может значительно влиять на результат работы алгоритмов определения главных направлений анизотропной породы в случаях, когда характерные направления деформации и анизотропии не совпадают. Изменение формы приводит к появлению неортогональности направлений поляризации дипольных мод, а также к зависимости этих направлений от частотного спектра распространяющейся волны. Таким образом, неортогональная модификация алгоритма Alford rotation позволяет выявить на этапе обработки данных признаки возможного несоответствия результата физическим свойствам породы.  

Применение частотных фильтров, сужающих спектр исходных данных измерений в низкочастотную область, позволяет получить более точные оценки главных направлений. Рассмотренные в работе примеры демонстрируют, что ортогональность направлений, полученных в ходе обработки, сама по себе не является критерием корректности результата - необходимо отсутствие сильной зависимости получаемого ответа от параметров фильтра и ширины временного окна. Недостатком обработки измерений в низкочастотной области является, очевидно, резкое падение энергии приходящего сигнала по сравнению с естественными шумами.

Результаты обработки каротажных измерений в быстрых породах ортогональным и неортогональным методами, основанными на диагонализации матрицы измерений, определяются поляризацией нормальных мод на высоких частотах. Данный вывод является достаточно интересным, так как, несмотря на продемонстрированную сильную частотную зависимость направлений собственных векторов дипольных мод, поперечная поляризация волн всего сигнала в целом существует и хорошо согласуется с высокочастотными решениями, полученными SAFE. 

%Важным результатом данной работы является оценка точности работы упомянутых алгоритмов, а также демонстрация возможного существования ортогонально поляризованных волн в быстрых породах,  не связанных однозначно с одним из факторов. Для дипольных мод на средних частотах существует область с достаточно резкой сменой направлений поляризации, которая однако не имеет определяющего значения для работы алгоритма, так как спектр используемых источников значительно шире.

Приведённый в статье материал демонстрирует возможности спектральных алгоритмов построенных на основе полуаналитического метода конечных элементов, для анализа и интерпретации особенностей волнового поля в скважинах. Высокая скорость расчётов, а также возможность распространения метода на среды с более общим типом анизотропии, среды с затуханием и преднапряженных пород, открывает широкие перспективы применения рассмотренной методики для улучшения качества обработки каротажных измерений. 

%\section{Краткое описание проблемы}

Модель распространения волн по скважине, используемой в алгоритмах обработки данных каротажных измерений, основана на уравнениях распространения плоской волны в анизотропной недисперсионной среде. Вектор значений $\mathbf{R}$ на приемниках может быть выражен через вектор возбуждения источника $\mathbf{S}$ в форме
$$
	\mathbf{R} = \mathbf{P}_{M \rightarrow R} \ \mathbf{D} \ \mathbf{P}_{S \rightarrow M} \ \mathbf{S},
$$
где $\mathbf{P}_{S \rightarrow M}$ - матрица, проецирующая вектор источника на главные направления распространения нормальных дипольных мод, $\mathbf{D}$ - матрица, определяющая распространение чистых мод вдоль скважины (в предположении, что моды не взаимодействуют друг с другом, считаем $\mathbf{D}$ диагональной), $\mathbf{P}_{M \rightarrow R}$ - матрица, проецирующая сигнал чистых мод на направления приемников.

В типовой схеме кросс-дипольных измерений с двумя ортогональными источниками, сонаправленными с осями $X$ и $Y$, $\mathbf{S}$ представляется через единичную матрицу, а данные с приемников могут быть записаны в форме матрицы данных:
$$
\left(
\begin{array}{cc}
XX & YX \\
XY & YY \\
\end{array}
\right) = \mathbf{R} = \mathbf{P}_{M \rightarrow R} \ \mathbf{D} \ \mathbf{P}_{S \rightarrow M}
$$
Если системы координат источников и приемников совпадают, то $\mathbf{P}_{M \rightarrow R}={\mathbf{P}_{S \rightarrow M}}^{T} = \mathbf{P}$, а матрица распространения чистых мод может быть выражена в форме
$$
	\mathbf{D} = \mathbf{P}^{-1} \ \mathbf{R} \ \mathbf{P}^{-T}.
$$
В традиционном варианте Alford rotation \cite{Alford1986} моды считаются ортогональными. В этом случае ортогональным поворотом на угол $\theta$ можно перейти в каноническую систему координат. В работе Dellinger et al. \cite{Dellinger1998} было показано, что даже если моды в скважине не являются ортогональными, $\mathbf{D}$ все равно может быть диагонализирована. Направление поляризации первой моды характеризуется углом $\theta$ относительно оси $X$, направление второй моды определяется поворотом на $\theta + \eta$ относительно оси $Y$.
\\

\begin{parcolumns}[colwidths={1=0.5\linewidth},rulebetween]{2}

\colchunk{
\textbf{Ортогональный Alford rotation}
\begin{align*}
\mathbf{P} & = \left(
\begin{array}{cc}
\cos \theta & -\sin \theta \\ 
\sin \theta & \cos \theta
\end{array} 
\right) \\
\mathbf{P}^{-1} & = \left(
\begin{array}{cc}
\cos \theta & \sin \theta \\ 
-\sin \theta & \cos \theta
\end{array} 
\right)
\end{align*}
}
\colchunk{
\textbf{Неортогональный Alford rotation}
\begin{align*}
\mathbf{P} &= \left(
\begin{array}{cc}
\cos \theta & -\sin (\theta+\eta) \\ 
\sin \theta & \cos (\theta+\eta)
\end{array} 
\right) \\
\mathbf{P}^{-1} &= \frac{1}{\cos \eta} \left(
\begin{array}{cc}
\cos (\theta+\eta) & \sin (\theta+\eta) \\ 
-\sin \theta & \cos (\theta)
\end{array} 
\right)
\end{align*}
}
\colplacechunks
\end{parcolumns}
%\begin{figure}[h]
\centering
\begin{minipage}{0.24\linewidth}
	\psfragfig[width=0.24\linewidth,crop=pdfcrop]{./images/SAFE/CS_15x10/cs15x10_sym_f1_5kHz+ang}		
\end{minipage}
\begin{minipage}{0.24\linewidth}
	\psfragfig[width=0.24\linewidth,crop=pdfcrop]{./images/SAFE/CS_15x10/cs15x10_sym_f2_39kHz+ang}		
\end{minipage}
\begin{minipage}{0.24\linewidth}
	\psfragfig[width=0.24\linewidth,crop=pdfcrop]{./images/SAFE/CS_15x10/cs15x10_sym_f3_28kHz+ang}		
\end{minipage}
\begin{minipage}{0.24\linewidth}
	\psfragfig[width=0.24\linewidth,crop=pdfcrop]{./images/SAFE/CS_15x10/cs15x10_sym_f5_52kHz+ang}		
\end{minipage}
\hfill
\begin{minipage}{0.24\linewidth}
	\psfragfig[width=0.24\linewidth,crop=pdfcrop]{./images/SAFE/CS_15x10/cs15x10_asym_f1_5kHz+ang}		
\end{minipage}
\begin{minipage}{0.24\linewidth}
	\psfragfig[width=0.24\linewidth,crop=pdfcrop]{./images/SAFE/CS_15x10/cs15x10_asym_f2_39kHz+ang}		
\end{minipage}
\begin{minipage}{0.24\linewidth}
	\psfragfig[width=0.24\linewidth,crop=pdfcrop]{./images/SAFE/CS_15x10/cs15x10_asym_f3_28kHz+ang}		
\end{minipage}
\begin{minipage}{0.24\linewidth}
	\psfragfig[width=0.24\linewidth,crop=pdfcrop]{./images/SAFE/CS_15x10/cs15x10_asym_f5_52kHz+ang}		
\end{minipage}
\hfill
\vspace{\baselineskip}
\begin{minipage}{0.24\linewidth}
	\psfragfig[width=0.24\linewidth,crop=pdfcrop]{./images/SAFE/CS_15x10/p_cs15x10_sym_f1_5kHz+ang}		
\end{minipage}
\begin{minipage}{0.24\linewidth}
	\psfragfig[width=0.24\linewidth,crop=pdfcrop]{./images/SAFE/CS_15x10/p_cs15x10_sym_f2_39kHz+ang}		
\end{minipage}
\begin{minipage}{0.24\linewidth}
	\psfragfig[width=0.24\linewidth,crop=pdfcrop]{./images/SAFE/CS_15x10/p_cs15x10_sym_f3_28kHz+ang}		
\end{minipage}
\begin{minipage}{0.24\linewidth}
	\psfragfig[width=0.24\linewidth,crop=pdfcrop]{./images/SAFE/CS_15x10/p_cs15x10_sym_f5_52kHz+ang}		
\end{minipage}
\hfill
\begin{minipage}{0.24\linewidth}
	\psfragfig[width=0.24\linewidth,crop=pdfcrop]{./images/SAFE/CS_15x10/p_cs15x10_asym_f1_5kHz+ang}		
\end{minipage}
\begin{minipage}{0.24\linewidth}
	\psfragfig[width=0.24\linewidth,crop=pdfcrop]{./images/SAFE/CS_15x10/p_cs15x10_asym_f2_39kHz+ang}		
\end{minipage}
\begin{minipage}{0.24\linewidth}
	\psfragfig[width=0.24\linewidth,crop=pdfcrop]{./images/SAFE/CS_15x10/p_cs15x10_asym_f3_28kHz+ang}		
\end{minipage}
\begin{minipage}{0.24\linewidth}
	\psfragfig[width=0.24\linewidth,crop=pdfcrop]{./images/SAFE/CS_15x10/p_cs15x10_asym_f5_52kHz+ang}		
\end{minipage}
\hfill

\caption{Графическое представление собственных векторов дипольных мод внутри (поле давления) и снаружи (компонента смещения z) скважины. Модель \ref{mnum: 5}. Здесь сплошные линии указывают направления, полученные неортогональным Alford rotation, примененным к исходным данным; прерывистая линия - к данным с оконной фильтрацией, прерывистая с точкой - к данным с низкочастотной фильтрацией.}
\end{figure}

\begin{figure}[h]
\centering
	\psfragfig[width=1\linewidth,crop=pdfcrop]{./images/SAFE/CS_15x10/modes_U+angles}		
	\caption{Представление значения $U = \sqrt{U_x^2+U_y^2+U_z^2}$ для собственных векторов в породе в зависимости от частоты. Модель \ref{mnum: 5}, быстрая порода. }
\end{figure}

\begin{figure}[h]
\centering
\begin{minipage}{0.24\linewidth}
	\psfragfig[width=0.24\linewidth,crop=pdfcrop]{./images/SAFE/AC_15x10/ac15x10_sym_f1_5kHz+ang}		
\end{minipage}
\begin{minipage}{0.24\linewidth}
	\psfragfig[width=0.24\linewidth,crop=pdfcrop]{./images/SAFE/AC_15x10/ac15x10_sym_f3_28kHz+ang}		
\end{minipage}
\begin{minipage}{0.24\linewidth}
	\psfragfig[width=0.24\linewidth,crop=pdfcrop]{./images/SAFE/AC_15x10/ac15x10_sym_f5_52kHz+ang}		
\end{minipage}
\begin{minipage}{0.24\linewidth}
	\psfragfig[width=0.24\linewidth,crop=pdfcrop]{./images/SAFE/AC_15x10/ac15x10_sym_f10_0kHz+ang}		
\end{minipage}
\hfill
\begin{minipage}{0.24\linewidth}
	\psfragfig[width=0.24\linewidth,crop=pdfcrop]{./images/SAFE/AC_15x10/ac15x10_asym_f1_5kHz+ang}		
\end{minipage}
\begin{minipage}{0.24\linewidth}
	\psfragfig[width=0.24\linewidth,crop=pdfcrop]{./images/SAFE/AC_15x10/ac15x10_asym_f3_28kHz+ang}		
\end{minipage}
\begin{minipage}{0.24\linewidth}
	\psfragfig[width=0.24\linewidth,crop=pdfcrop]{./images/SAFE/AC_15x10/ac15x10_asym_f5_52kHz+ang}		
\end{minipage}
\begin{minipage}{0.24\linewidth}
	\psfragfig[width=0.24\linewidth,crop=pdfcrop]{./images/SAFE/AC_15x10/ac15x10_asym_f10_0kHz+ang}		
\end{minipage}
\hfill
\vspace{\baselineskip}
\begin{minipage}{0.24\linewidth}
	\psfragfig[width=0.24\linewidth,crop=pdfcrop]{./images/SAFE/AC_15x10/p_ac15x10_sym_f1_5kHz+ang}		
\end{minipage}
\begin{minipage}{0.24\linewidth}
	\psfragfig[width=0.24\linewidth,crop=pdfcrop]{./images/SAFE/AC_15x10/p_ac15x10_sym_f3_28kHz+ang}		
\end{minipage}
\begin{minipage}{0.24\linewidth}
	\psfragfig[width=0.24\linewidth,crop=pdfcrop]{./images/SAFE/AC_15x10/p_ac15x10_sym_f5_52kHz+ang}		
\end{minipage}
\begin{minipage}{0.24\linewidth}
	\psfragfig[width=0.24\linewidth,crop=pdfcrop]{./images/SAFE/AC_15x10/p_ac15x10_sym_f10_0kHz+ang}		
\end{minipage}
\hfill
\begin{minipage}{0.24\linewidth}
	\psfragfig[width=0.24\linewidth,crop=pdfcrop]{./images/SAFE/AC_15x10/p_ac15x10_asym_f1_5kHz+ang}		
\end{minipage}
\begin{minipage}{0.24\linewidth}
	\psfragfig[width=0.24\linewidth,crop=pdfcrop]{./images/SAFE/AC_15x10/p_ac15x10_asym_f3_28kHz+ang}		
\end{minipage}
\begin{minipage}{0.24\linewidth}
	\psfragfig[width=0.24\linewidth,crop=pdfcrop]{./images/SAFE/AC_15x10/p_ac15x10_asym_f5_52kHz+ang}		
\end{minipage}
\begin{minipage}{0.24\linewidth}
	\psfragfig[width=0.24\linewidth,crop=pdfcrop]{./images/SAFE/AC_15x10/p_ac15x10_asym_f10_0kHz+ang}		
\end{minipage}
\hfill

\caption{Графическое представление собственных векторов дипольных мод внутри (поле давления) и снаружи (компонента смещения z) скважины. Модель \ref{mnum: 3}. Здесь сплошные линии указывают направления, полученные неортогональным Alford rotation, примененным к исходным данным; прерывистая линия - к данным с низкочастотной фильтрацией, прерывистая с точкой - к данным с высокочастотной фильтрацией.}
\end{figure}

\begin{figure}[h]
\centering
	\psfragfig[width=1\linewidth,crop=pdfcrop]{./images/SAFE/AC_15x10/ac15x10_modes_U+angles}
	\caption{Представление значения $U = \sqrt{U_x^2+U_y^2+U_z^2}$ для собственных векторов в породе в зависимости от частоты. Модель \ref{mnum: 3}, медленная порода. }		
\end{figure}

\textbf{Зависимость энергии диагональных компонент от значений $\theta$ и $\eta$}\\
			\psfragfig[width=0.40\linewidth,crop=pdfcrop]{./images/nonorth_alford/solution_min_gs_rot4c}
			\label{fig:rot4_gs_solution}

\begin{minipage}[h]{0.47\linewidth}
\begin{center}
\textbf{Результат работы TKO}
			\psfragfig[width=0.40\linewidth,crop=pdfcrop]{./images/nonorth_alford/el20x10_TTI60_TKO_compare}\\
	  		\label{fig:rot4_tko_comp}
\end{center}	  		
\end{minipage}

\begin{figure}[h]
\centering
\begin{minipage}{0.47\linewidth}
	\psfragfig[width=0.47\linewidth,crop=pdfcrop]{./images/SAFE/CS_15x10/cs15x10_sym_f1_5kHz_U+ang}		
\end{minipage}
\hfill
\begin{minipage}{0.47\linewidth}
	\psfragfig[width=0.47\linewidth,crop=pdfcrop]{./images/SAFE/CS_15x10/cs15x10_sym_f7_76kHz_U+ang}		
\end{minipage}
\vspace{\baselineskip}
\begin{minipage}{0.47\linewidth}
	\psfragfig[width=0.47\linewidth,crop=pdfcrop]{./images/SAFE/CS_15x10/cs15x10_asym_f1_5kHz_U+ang}		
\end{minipage}
\hfill
\begin{minipage}{0.47\linewidth}
	\psfragfig[width=0.47\linewidth,crop=pdfcrop]{./images/SAFE/CS_15x10/cs15x10_asym_f7_76kHz_U+ang}		
\end{minipage}
\caption{Представление значения $|U|$ для собственных векторов в породе в зависимости от частоты. Модель \ref{mnum: 5}, быстрая порода. Здесь сплошные линии указывают направления, полученные неортогональным Alford rotation, примененным к исходным данным; прерывистая линия - к данным с оконной фильтрацией, прерывистая с точкой - к данным с низкочастотной фильтрацией. }	
\end{figure}

\begin{figure}[h]
\begin{minipage}{1\linewidth}
	\psfragfig[width=1\linewidth,crop=pdfcrop]{./images/SAFE/CS_15x10/disp_close_spec}		
\end{minipage}
\vspace{\baselineskip}
\begin{minipage}{1\linewidth}
	\psfragfig[width=1\linewidth,crop=pdfcrop]{./images/SAFE/CS_15x10/modes_close_spec}		
\end{minipage}
\end{figure}
\clearpage

%\bibliography{./library/library}
%\include{include/var_bibliography_link}
\bibliography{c:/Users/German/Documents/TeX_Library/library}
%\bibliography{d:/Documents/Workfiles/Literature/TeX_Library/library}

\bibliographystyle{unsrt}
%\bibliographystyle{plainnat_no_url}

%\appendix
%\section{Фильтрация данных}
%
%%\begin{minipage}[c]{0.47\linewidth}	
%%\begin{center}
%%		Ортогональный Alford rotation \\
%%		\psfragfig[width=0.40\linewidth,crop=pdfcrop]{./images/nonorth_alford/el20x10_TTI60_rot4c_scheme}			
%%	  		\label{fig:rot4_scheme}
%%\end{center}	  		
%%\end{minipage} \hfill
%%\begin{minipage}[c]{0.47\linewidth}
%%\begin{center}
%%		Неортогональный Alford rotation\\
%%			\psfragfig[width=0.40\linewidth,crop=pdfcrop]{./images/nonorth_alford/el20x10_TTI60_gs_rot4c_scheme}
%%			\label{fig:rot4_gs_scheme}
%%\end{center}
%%\end{minipage} 	\\
%\begin{table}[H]
%\footnotesize
%%\centering
%\caption{Параметры применяемых фильтров}
%\renewcommand{\arraystretch}{1.5}
%\textbf{Низкочастотные фильтры} \\
%\begin{tabularx}{\textwidth}{|C|c|c|c|c|}
%\hline Обозначение & $A_{pass}, $ дБ & $A_{stop}, $ дБ & $F_{pass}, $ Гц & $F_{stop}, $ Гц \\ 
%\hline \textbf{}\lffiltnum{\label{lffnum: 3}} & 1 & 80 & 3000 & 4000 \\ 
%\hline \textbf{}\lffiltnum{\label{lffnum: 4}} & 1 & 80 & 3000 & 3500 \\ 
%\hline 
%\end{tabularx} \\
%\textbf{Высокочастотные фильтры} \\
%\begin{tabularx}{\textwidth}{|C|c|c|c|c|}
%\hline Обозначение & $A_{pass}, $ дБ & $A_{stop}, $ дБ & $F_{pass}, $ Гц & $F_{stop}, $ Гц \\ \hline
%\hline \textbf{}\hffiltnum{\label{hffnum: 1}} & 1 & 80 & 5000 & 4000 \\ 
%\hline \textbf{}\hffiltnum{\label{hffnum: 2}} & 1 & 80 & 6000 & 5000 \\ 
%\hline \textbf{}\hffiltnum{\label{hffnum: 3}} & 1 & 80 & 7000 & 6000 \\ 
%\hline 
%\end{tabularx}
%\renewcommand{\arraystretch}{1.0}
%\end{table}

\end{document} 