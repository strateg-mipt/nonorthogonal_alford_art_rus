% !TeX spellcheck = ru_RU
\documentclass[a4paper,11pt]{article}
\usepackage[process=auto]{pstool}
\usepackage[english,russian]{babel}
\usepackage[T2A]{fontenc}
\usepackage[utf8]{inputenc}
\usepackage{amssymb,amsmath}
\usepackage{gensymb,textcomp,latexsym}
\usepackage{graphicx}
\usepackage{tabularx}
\usepackage[pdftex, left=1in, right=1in, top=1in, bottom=2cm]{geometry}
\usepackage{parcolumns}
\usepackage{multirow}
\usepackage{tikz}

%\usepackage[usenames,dvipsnames]{xcolor}
%\usepackage[font=small,labelfont=bf]{caption}
\usepackage[center]{subfigure}
\renewcommand{\thesubfigure}{(\asbuk{subfigure})~}
\newcommand{\figref}[1]{Рис.~\ref{#1}}
\usepackage[pdfauthor={Shchelik},pdftitle={Nonorthogonal Alford test},pdfstartview=XYZ,bookmarks=true,colorlinks=true,linkcolor=blue,urlcolor=blue,citecolor=blue,
bookmarks=true,linktocpage=true,hyperindex=true]{hyperref}
%\usepackage[hyperpageref]{backref}

%\usepackage[section]{placeins}
%\usepackage{graphicx}
%\usepackage{epsfig}
%\usepackage{epstopdf}
%\usepackage{subfigure}
\usepackage{float}
%\usepackage{booktabs}
%\graphicspath{ {./image_test/} }

\newcounter{modelnum}
\newcommand{\modelnum}[1]{\refstepcounter{modelnum}Модель \themodelnum #1}

%Filters counters
\newcounter{wfiltnum}
\newcommand{\wfiltnum}[1]{\refstepcounter{wfiltnum}ОФ-\thewfiltnum #1}
\newcounter{lffiltnum}
\newcommand{\lffiltnum}[1]{\refstepcounter{lffiltnum}НЧФ-\thelffiltnum #1}
\newcounter{hffiltnum}
\newcommand{\hffiltnum}[1]{\refstepcounter{hffiltnum}ВЧФ-\thehffiltnum #1}
\newcounter{bffiltnum}
\newcommand{\bffiltnum}[1]{\refstepcounter{bffiltnum}ПЧФ-\thebffiltnum #1}
%tabularx options
\newcolumntype{C}{>{\centering}X}


\begin{document}
\part*{Исходные данные}
\today

\section{Описание моделей}
%Параметры быстрой породы, $\rho = 2640$ кг/м$^3$, $V_P = 5219$ м/с, $V_S = 3257$ м/с.
\begin{table}[H]
\footnotesize
\centering
\caption{Параметры упругих анизотропных материалов}
\renewcommand{\arraystretch}{1.5}
\begin{tabularx}{\textwidth}{|C|c|c|c|c|c|c|c|}
\hline \multirow{2}{*}{Название}  & Плотность & \multicolumn{6}{c|}{Упругие модули, ГПа} \\ 
\cline{3-8}  & кг/м$^3$ & $C_{11}$ & $C_{12}$ & $C_{13}$ & $C_{33}$ & $C_{44}$ & $C_{66}$ \\ \hline
\hline Cotton Valey Shale & 2640 & 74.73 & 14.75 & 25.29 & 58.84 & 22.05 & 29.99 \\ 
\hline Bakken Shale & 2230 & 40.9 & 10.3 & 8.5 & 26.9 & 10.5 & 15.3 \\ 
\hline Austin Chalk & 2200 & 22.0 & 15.8 & 12 & 14 & 2.4 & 3.1 \\ 
\hline 
\end{tabularx} 
\label{tab:properties}
\renewcommand{\arraystretch}{1.0}
\end{table}

\begin{table}[h]
\footnotesize
\centering
\caption{Параметры модельных задач}
\renewcommand{\arraystretch}{1.5}
\begin{tabularx}{\textwidth}{|C|l|c|l|c|c|}
\hline  Обозначение & Форма скважины & Геометрия, см & Материал породы & TI угол $\zeta$ & TI угол $\phi$ \\ \hline
\hline \textbf{}\modelnum{\label{mnum: 1}}& Эллиптическая & $12.70 \times 10.16$ & Bakken Shale & 90 & 45 \\
\hline \textbf{}\modelnum{\label{mnum: 2}}& Эллиптическая & $15.00 \times 10.00$ & Bakken Shale & 90 & 45 \\
\hline \textbf{}\modelnum{\label{mnum: 3}}& Эллиптическая & $12.70 \times 10.16$ & Cotton Valey Shale & 90 & 45 \\ 
\hline \textbf{}\modelnum{\label{mnum: 4}}& Эллиптическая & $15.00 \times 10.00$ & Cotton Valey Shale & 90 & 45 \\
\hline \textbf{}\modelnum{\label{mnum: 5}}& Эллиптическая & $12.70 \times 10.16$ & Bakken Shale & 90 & 68.5 \\
\hline \textbf{}\modelnum{\label{mnum: 6}}& Эллиптическая & $12.70 \times 10.16$ & Bakken Shale & 90 & 23.5 \\
\hline \textbf{}\modelnum{\label{mnum: 15}}& Эллиптическая & $20.00 \times 10.00$ & Austin Chalk & 60 & 45 \\
\hline 
\end{tabularx}
\label{tab:models_description}
\renewcommand{\arraystretch}{1.0}
\end{table}

\section{Обработка Alford rotation}

\begin{table}[h]
\footnotesize
\centering
\caption{Результаты расчетов}
\renewcommand{\arraystretch}{1.5}
\begin{tabularx}{\textwidth}{C|rr|rr|r|rr}
				&\multicolumn{1}{c}{$\theta_1^o$} & \multicolumn{1}{c|}{$\theta_1^n$} & \multicolumn{1}{c}{$\theta_2^o$} & \multicolumn{1}{c|}{$\theta_2^n$} & \multicolumn{1}{c|}{$\Delta\theta^n$}& \multicolumn{1}{c}{$E_{cr}^o/E_{t}^o$, \%} & \multicolumn{1}{c}{$E_{cr}^n/E_{t}^n$, \%} \\ \hline
\hline Модель \ref{mnum: 1} & 15.56 & 14.71 &  -74.44 & -73.92  & 1.36  & 2.99 & 2.98 \\
 	   Модель \ref{mnum: 2} & 8.39 & 8.12 & -81.61 & -81.56 & 0.3 & 1.69 & 1.69 \\
 	   Модель \ref{mnum: 4} & 1.63 & 1.78 & -88.37  & -88.42  & 0.0  & 0.64 & 0.64 \\	   
	   Модель \ref{mnum: 3} & 3.26 & 3.04 & -86.74  & -86.65  & 0.3 & 0.75 & 0.75 \\ 	 
	   Модель \ref{mnum: 5} & 15.72 & 14.01 & -74.28  & -72.90  & 3.09 & 5.17 & 5.09 \\ 	
	   Модель \ref{mnum: 6} & 8.97 & 7.36 & -81.03  & -80.60  & 2.04 & 1.22 & 1.15 \\
	   Модель \ref{mnum: 15} & -10.51 & -26.36 & 79.49  & 80.46  & 16.82 & 17.26 & 15.11 \\
 	   \hline
\end{tabularx} 
\label{tab:std_process_results}
\renewcommand{\arraystretch}{1.0}
\end{table}

\section{Обработка Alford rotation с фильтрами}

\begin{table}[H]
\footnotesize
%\centering
\caption{Параметры применяемых фильтров}
\renewcommand{\arraystretch}{1.5}
\textbf{Оконные фильтры} \\
\begin{tabularx}{\textwidth}{|C|c|c|}
\hline Обозначение & Величина окна, мс & Интервальное время, мкс/м  \\ \hline
\hline \textbf{}\wfiltnum{\label{wfnum: 3}} & 0.5 & 320  \\
\hline \textbf{}\wfiltnum{\label{wfnum: 4}} & 0.5 & 650  \\
\hline \textbf{}\wfiltnum{\label{wfnum: 5}} & 0.6 & 320  \\
\hline \textbf{}\wfiltnum{\label{wfnum: 6}} & 0.5 & 400  \\
\hline 
\end{tabularx} \\
\textbf{Низкочастотные фильтры} \\
\begin{tabularx}{\textwidth}{|C|c|c|c|c|}
\hline Обозначение & $A_{pass}, $ дБ & $A_{stop}, $ дБ & $F_{pass}, $ Гц & $F_{stop}, $ Гц \\ \hline
\hline \textbf{}\lffiltnum{\label{lffnum: 1}} & 1 & 80 & 4000 & 5000 \\ 
\hline \textbf{}\lffiltnum{\label{lffnum: 3}} & 1 & 80 & 3000 & 4000 \\ 
\hline \textbf{}\lffiltnum{\label{lffnum: 4}} & 1 & 80 & 3000 & 3500 \\ 
\hline \textbf{}\lffiltnum{\label{lffnum: 5}} & 1 & 60 & 2500 & 3500 \\ 
\hline \textbf{}\lffiltnum{\label{lffnum: 6}} & 1 & 60 & 2000 & 3000 \\ 
\hline 
\end{tabularx} \\
\textbf{Высокочастотные фильтры} \\
\begin{tabularx}{\textwidth}{|C|c|c|c|c|}
\hline Обозначение & $A_{pass}, $ дБ & $A_{stop}, $ дБ & $F_{pass}, $ Гц & $F_{stop}, $ Гц \\ \hline
\hline \textbf{}\hffiltnum{\label{hffnum: 1}} & 1 & 80 & 5000 & 4000 \\ 
\hline \textbf{}\hffiltnum{\label{hffnum: 2}} & 1 & 80 & 6000 & 5000 \\ 
\hline \textbf{}\hffiltnum{\label{hffnum: 3}} & 1 & 80 & 7000 & 6000 \\ 
\hline 
\end{tabularx}
\textbf{Полосовые фильтры} \\
\begin{tabularx}{\textwidth}{|C|c|c|c|c|l|l|}
	\hline Обозначение & $A_{pass}, $ дБ & $A_{stop}, $ дБ & $F_{stop}^1, $ Гц & $F_{pass}^1, $ Гц & $F_{pass}^2, $ Гц & $F_{stop}^2, $ Гц\\ \hline
	\hline \textbf{}\bffiltnum{\label{bffnum: 1}} & 1 & 80 & 5000 & 4000 & & \\ 
	\hline \textbf{}\bffiltnum{\label{bffnum: 2}} & 1 & 80 & 6000 & 5000 & & \\ 
	\hline \textbf{}\bffiltnum{\label{bffnum: 3}} & 1 & 80 & 7000 & 6000 & & \\ 
	\hline 
\end{tabularx}
\renewcommand{\arraystretch}{1.0}
\end{table}

\begin{figure}[h]
\centering
\begin{minipage}{0.49\linewidth}
	\centering \textbf{Схема низкочастотного фильтра}
	\psfragfig[width=0.49\linewidth,crop=pdfcrop]{./images/filter_schemes/filter_scheme_lp} \\
\end{minipage}
\begin{minipage}{0.49\linewidth}
	\centering \textbf{Схема высокочастотного фильтра}
	\psfragfig[width=0.49\linewidth,crop=pdfcrop]{./images/filter_schemes/filter_scheme_hp} \\
\end{minipage}
\begin{minipage}{0.49\linewidth}
	\centering \textbf{Схема полосового фильтра}
	\psfragfig[width=0.49\linewidth,crop=pdfcrop]{./images/filter_schemes/filter_scheme_bp} \\
\end{minipage}
\begin{minipage}{0.49\linewidth}
	\centering \textbf{Схема полосового фильтра}
	\psfragfig[width=0.49\linewidth,crop=pdfcrop]{./images/filter_schemes/filter_scheme_bp_ext} \\
\end{minipage}
\caption{Используемые обозначения параметров фильтров. }
\label{fig:filter_schemes_all}
\end{figure}

\begin{table}[h]
\footnotesize
\centering
\caption{Результаты расчетов с применением фильтров}
\renewcommand{\arraystretch}{1.5}
\begin{tabularx}{\textwidth}{X|rr|rr|r|ll}
				&\multicolumn{1}{c}{$\theta_1^o$} & \multicolumn{1}{c|}{$\theta_1^n$} & \multicolumn{1}{c}{$\theta_2^o$} & \multicolumn{1}{c|}{$\theta_2^n$} & \multicolumn{1}{c|}{$\Delta\theta^n$}& \multicolumn{1}{c}{$E_{cr}^o/E_{t}^o$} & \multicolumn{1}{c}{$E_{cr}^n/E_{t}^n$} \\ \hline
\hline	\textbf{Модель \ref{mnum: 1}} & \textbf{15.5623} & \textbf{14.7154} & \textbf{-74.4377}  & \textbf{-73.923}  & \textbf{1.36}  & \textbf{0.0299} & \textbf{0.0297} \\
		Модель \ref{mnum: 1} с ОФ \ref{wfnum: 5} & 45.6193 & 63.3189 & -44.3807 & -49.2086  & 22.5  & 0.1198 & 0.0617 \\
		Модель \ref{mnum: 1} с НЧФ \ref{lffnum: 3} & 49.4110 & 40.4599 & -40.5890 & -35.4091  & 14.1 & 0.0204 & 0.0124\\
		Модель \ref{mnum: 1} с ВЧФ \ref{hffnum: 1} & 14.0073 & 13.2547 & -75.9927 & -75.5753  & 1.2 & 0.0039 & 0.0038\\
\hline	\textbf{Модель \ref{mnum: 2}} & \textbf{8.3898} & \textbf{8.1241} & \textbf{-81.6102}  & \textbf{-81.5622} & \textbf{0.3}  & \textbf{0.0169} & \textbf{0.0169} \\
		Модель \ref{mnum: 2} с ОФ \ref{wfnum: 6} & 54.7210 & 68.6300 & -35.2790 & -46.1632  & 24.8  & 0.2115 & 0.1219 \\
		Модель \ref{mnum: 2} с НЧФ \ref{lffnum: 3} & 41.2105 & 25.7145 & -48.7895 & -31.9696  & 32.3 & 0.2210 & 0.1199\\
		Модель \ref{mnum: 4} с НЧФ \ref{lffnum: 5} & 52.1713 & 32.5122 & --37.8287 & -27.8662  & 29.62  & 0.1230 & 0.0638 \\
		Модель \ref{mnum: 2} с НЧФ \ref{lffnum: 6} & 50.9972 & 48.9931 & -39.0028 & -37.8245  & 3.18 & 0.0668 & 0.0658\\
		Модель \ref{mnum: 2} с ВЧФ \ref{hffnum: 2} & 7.5375 & 10.2188 & -82.4625 & -82.6270  & 2.84 & 0.0110 & 0.0101\\
\hline	\textbf{Модель \ref{mnum: 4}} & \textbf{1.6281} & \textbf{1.7796} & \textbf{-88.3719}  & \textbf{-88.4215}  & \textbf{0.0}  & \textbf{0.0064} & \textbf{0.0064} \\
		Модель \ref{mnum: 4} с ОФ \ref{wfnum: 3} & 49.405 & 58.614 & -40.595 & -54.798  & 23.4  & 0.1297 & 0.0691 \\
		Модель \ref{mnum: 4} с ОФ \ref{wfnum: 4} & 2.5785 & 2.9429 & -87.422 & -87.524  & 0.46  & 0.0088 & 0.0088 \\
		Модель \ref{mnum: 4} с НЧФ \ref{lffnum: 1} & 2.4372 & 2.1617 & -87.5628 & -74.4739  & 13.4  & 0.0132 & 0.0098 \\
		Модель \ref{mnum: 4} с НЧФ \ref{lffnum: 4} & 6.0268 & 7.7323 & -83.9732 & -56.5965  & 25.6  & 0.0789 & 0.0726 \\
		Модель \ref{mnum: 4} с ВЧФ \ref{hffnum: 3} & 1.4914 & 2.0357 & -88.5086 & -88.7084  & 0.74  & 0.0184 & 0.0183\\
\hline	\textbf{Модель \ref{mnum: 3}} & \textbf{3.2568} & \textbf{3.04304} & \textbf{-86.7431}  & \textbf{-86.6553}  & \textbf{0.3}  & \textbf{0.0075} & \textbf{0.0075} \\
		Модель \ref{mnum: 3} с НЧФ \ref{lffnum: 4} & 48.417 & 39.731 & -41.583 & -35.501  & 14.8  & 0.0974 & 0.0701 \\
		Модель \ref{mnum: 3} с ВЧФ \ref{hffnum: 3} & 2.7739 & 3.2596 & -87.2261 & -87.4317  & 0.69  & 0.0045 & 0.0045\\		
\hline	\textbf{Модель \ref{mnum: 5}} & \textbf{15.72} & \textbf{14.01} & \textbf{-74.28}  & \textbf{-72.8976}  & \textbf{3.09}  & \textbf{0.0517} & \textbf{0.0509} \\
		Модель \ref{mnum: 5} с НЧФ \ref{lffnum: 1} & 55.702 & 36.6182 & -34.2982 & -28.5772  & 24.8  & 0.2122 & 0.1453 \\
		Модель \ref{mnum: 5} с НЧФ \ref{lffnum: 5} & 70.5594 & 67.1189 & -19.4406 & -17.8821  & 5.0  & 0.0180 & 0.0172 \\
		Модель \ref{mnum: 5} с ВЧФ \ref{hffnum: 2} & 14.2617 & 14.8978 & -75.7383 & -76.1548  & 1.0526  & 0.0074 & 0.0073 \\
\hline	\textbf{Модель \ref{mnum: 6}} & \textbf{8.9734} & \textbf{7.3612} & \textbf{-81.03}  & \textbf{-80.60}  & \textbf{2.04}  & \textbf{0.01219} & \textbf{0.0115} \\
		Модель \ref{mnum: 6} с НЧФ \ref{lffnum: 1} & 20.2834 & 15.8551 & -69.7166 & -60.8906  & 13.25  & 0.0594 & 0.0274 \\
		Модель \ref{mnum: 6} с ВЧФ \ref{hffnum: 2} & 7.6511 & 8.3492 & -82.3489 & -82.5229  & 0.8721 & 0.00136 & 0.001267 \\	
\hline	\textbf{Модель \ref{mnum: 15}} & \textbf{-10.5111} & \textbf{-26.3606} & \textbf{79.4889}  & \textbf{-80.4622}  & \textbf{16.82}  & \textbf{0.1726} & \textbf{0.1511} \\
		Модель \ref{mnum: 15} с НЧФ \ref{lffnum: 1} & -30.6596 & -41.5868 & 59.3404 & 68.1357  & 19.7225  & 0.0358 & 0.0027 \\
		Модель \ref{mnum: 15} с ВЧФ \ref{hffnum: 1} & -1.7554 & -4.9971 & 88.2446 & 88.5870  & 3.5841 & 0.0602 & 0.0597 \\	
\hline	

\end{tabularx} 
\label{tab:filter_process_results}
\renewcommand{\arraystretch}{1.0}
\end{table}

\begin{figure}[h]
	\centering
	\begin{minipage}{0.98\linewidth}
		\centering %\textbf{Обработка полосовым фильтром}
		\psfragfig[width=0.80\linewidth,crop=pdfcrop]{./images/filtering/BS_HTI_45/10x8in/graph_bp_filter_study_500} \\
	\end{minipage}
	\caption{Обработка полосовым фильтром Модели \ref{mnum: 1}}
	\label{fig:bp_filter_graph_bs_hti_45_10x8}
\end{figure}
\begin{figure}[h]
	\centering
	\begin{minipage}{0.98\linewidth}
		\centering %\textbf{Обработка полосовым фильтром}
		\psfragfig[width=0.80\linewidth,crop=pdfcrop]{./images/filtering/BS_HTI_45/10x8in/graph_SAFE_eigenvectors_angles} \\
	\end{minipage}
	\caption{Ориентация собственных векторов дипольных мод для Модели \ref{mnum: 2}}
	\label{fig:eigenvec_ang_graph_bs_hti_45_10x8}
\end{figure}

\begin{figure}[h]
	\centering
	\begin{minipage}{0.98\linewidth}
		\centering %\textbf{Обработка полосовым фильтром}
		\psfragfig[width=0.80\linewidth,crop=pdfcrop]{./images/filtering/BS_HTI_45/15x10/graph_bp_filter_study_500} \\
	\end{minipage}
	\caption{Обработка полосовым фильтром Модели \ref{mnum: 1}}
	\label{fig:bp_filter_graph_bs_hti_45_15x10}
\end{figure}
\begin{figure}[h]
	\centering
	\begin{minipage}{0.98\linewidth}
		\centering %\textbf{Обработка полосовым фильтром}
		\psfragfig[width=0.80\linewidth,crop=pdfcrop]{./images/filtering/BS_HTI_45/15x10/graph_SAFE_eigenvectors_angles} \\
	\end{minipage}
	\caption{Ориентация собственных векторов дипольных мод для Модели \ref{mnum: 2}}
	\label{fig:eigenvec_ang_graph_bs_hti_45_15x10}
\end{figure}


\begin{figure}[h]
	\centering
	\begin{minipage}{0.98\linewidth}
		\centering %\textbf{Обработка полосовым фильтром}
		\psfragfig[width=0.80\linewidth,crop=pdfcrop]{./images/filtering/BS_HTI_45/10x8in/graph_bp_filter_study_500_highfreq} \\
	\end{minipage}
	\caption{Обработка полосовым фильтром Модели \ref{mnum: 1} на высоких частотах}
	\label{fig:bp_filter_high_graph_bs_hti_45_10x8}
\end{figure}
\begin{figure}[h]
	\centering
	\begin{minipage}{0.98\linewidth}
		\centering %\textbf{Обработка полосовым фильтром}
		\psfragfig[width=0.80\linewidth,crop=pdfcrop]{./images/filtering/BS_HTI_45/10x8in/graph_bh_ell_6kHz_angles} \\
	\end{minipage}
	\caption{Ориентация собственных векторов дипольных мод на частоте 6 кГц для Модели \ref{mnum: 1} в зависимости от степени эллиптичности ствола скважины.}
	\label{fig:ell_dep_eigenvec_ang_graph_bs_hti_45_10x8}
\end{figure}

%\section{Сравнение с SAFE}
%
%На рисунках \ref{fig:bs10_8_HTI45} и \ref{fig:bs15_10_HTI45} приведена визуализация амплитуды давления собственных векторов, относящихся к двум дипольным модам внутри скважины. Направление градиента отражает ориентацию поляризации каждой из волн. Как можно заметить, поляризация обеих моды обладает заметной зависимостью от частоты. Заметим также, что поляризация на низких частотах практически не отличается для случаев с эллиптичностью 25\% и 50\%. Неортогональность этих направлений имеет наиболее сильный характер в диапазоне частот от 2 до 4 кГц для рассматриваемых случаев. На полученные профили амплитуд нанесены результаты обработки данных из таблицы \ref{tab:filter_process_results}: для частот 1.94 кГц и 3.29 кГц нанесены результаты с применением низкочастотной фильтрации и без, для частот 5.52 кГц и 8.65 кГц - высокочастотной фильтрации и без неё. 

%\begin{figure}[h]
%\centering
%\begin{minipage}{0.49\linewidth}
%	\centering \textbf{Модель \ref{mnum: 1}}
%	\psfragfig[width=0.49\linewidth,crop=pdfcrop]{./images/nonorth_alford/el10x8_inch_HTI_BS_f45_disp_modes+SAFE} \\
%\end{minipage}
%\begin{minipage}{0.49\linewidth}
%	\centering \textbf{Модель \ref{mnum: 2}}
%	\psfragfig[width=0.49\linewidth,crop=pdfcrop]{./images/nonorth_alford/el15x10_HTI_BS_f45_disp_modes+SAFE} \\
%\end{minipage}
%\begin{minipage}{0.49\linewidth}
%	\centering \textbf{Модель \ref{mnum: 3}}
%	\psfragfig[width=0.49\linewidth,crop=pdfcrop]{./images/nonorth_alford/el10x8_inch_HTI_CS_f45_disp_modes+SAFE} \\
%\end{minipage}
%\begin{minipage}{0.49\linewidth}
%	\centering \textbf{Модель \ref{mnum: 4}}
%	\psfragfig[width=0.49\linewidth,crop=pdfcrop]{./images/nonorth_alford/el15x10_HTI_CS_f45_disp_modes+SAFE_new} \\
%\end{minipage}
%\caption{Дисперсионные кривые для основных задач. }
%\label{fig:disp_curves_all}
%\end{figure}
%
%\begin{figure}[h]
%\centering
%\begin{minipage}{0.24\linewidth}
%	\psfragfig[width=0.24\linewidth,crop=pdfcrop]{./images/SAFE/SAFE_BS_10x8_HTI_45/P_s_2kHz}		
%\end{minipage}
%\begin{minipage}{0.24\linewidth}
%	\psfragfig[width=0.24\linewidth,crop=pdfcrop]{./images/SAFE/SAFE_BS_10x8_HTI_45/P_s_3_3kHz}		
%\end{minipage}
%\begin{minipage}{0.24\linewidth}
%	\psfragfig[width=0.24\linewidth,crop=pdfcrop]{./images/SAFE/SAFE_BS_10x8_HTI_45/P_s_5_5kHz}		
%\end{minipage}
%\begin{minipage}{0.24\linewidth}
%	\psfragfig[width=0.24\linewidth,crop=pdfcrop]{./images/SAFE/SAFE_BS_10x8_HTI_45/P_s_8_6kHz}		
%\end{minipage}
%\begin{minipage}{0.24\linewidth}
%	\psfragfig[width=0.24\linewidth,crop=pdfcrop]{./images/SAFE/SAFE_BS_10x8_HTI_45/P_a_2kHz}		
%\end{minipage}
%\begin{minipage}{0.24\linewidth}
%	\psfragfig[width=0.24\linewidth,crop=pdfcrop]{./images/SAFE/SAFE_BS_10x8_HTI_45/P_a_3_3kHz}		
%\end{minipage}
%\begin{minipage}{0.24\linewidth}
%	\psfragfig[width=0.24\linewidth,crop=pdfcrop]{./images/SAFE/SAFE_BS_10x8_HTI_45/P_a_5_5kHz}		
%\end{minipage}
%\begin{minipage}{0.24\linewidth}
%	\psfragfig[width=0.24\linewidth,crop=pdfcrop]{./images/SAFE/SAFE_BS_10x8_HTI_45/P_a_8_6kHz}		
%\end{minipage}
%\vspace{\baselineskip}
%\renewcommand{\arraystretch}{1.0}
%\footnotesize
%\begin{tabular*}{\textwidth}{@{\extracolsep{\fill} }clclcl}
%\tikz \draw (0,0) -- (1cm,0); & ортогональный & \tikz \draw[dashed] (0,0) -- (1cm,0); & ортогональный & \tikz \draw[dashdotted] (0,0) -- (1cm,0); & неортогональный    \\
% & без фильтрации & & с фильтрацией & & с фильтрацией \\
%\end{tabular*}
%\renewcommand{\arraystretch}{1.0}
%\normalsize
%\caption{Результаты расчетов собственных векторов для Модели \ref{mnum: 1}}
%\label{fig:bs10_8_HTI45}
%\end{figure}

%\begin{figure}[h]
%\centering
%\begin{minipage}{0.24\linewidth}
%	\psfragfig[width=0.24\linewidth,crop=pdfcrop]{./images/SAFE/SAFE_BS_15x10_HTI_45/P_s_2kHz}		
%\end{minipage}
%\begin{minipage}{0.24\linewidth}
%	\psfragfig[width=0.24\linewidth,crop=pdfcrop]{./images/SAFE/SAFE_BS_15x10_HTI_45/P_s_3_3kHz}		
%\end{minipage}
%\begin{minipage}{0.24\linewidth}
%	\psfragfig[width=0.24\linewidth,crop=pdfcrop]{./images/SAFE/SAFE_BS_15x10_HTI_45/P_s_5_5kHz}		
%\end{minipage}
%\begin{minipage}{0.24\linewidth}
%	\psfragfig[width=0.24\linewidth,crop=pdfcrop]{./images/SAFE/SAFE_BS_15x10_HTI_45/P_s_8_6kHz}		
%\end{minipage}
%\begin{minipage}{0.24\linewidth}
%	\psfragfig[width=0.24\linewidth,crop=pdfcrop]{./images/SAFE/SAFE_BS_15x10_HTI_45/P_a_2kHz}		
%\end{minipage}
%\begin{minipage}{0.24\linewidth}
%	\psfragfig[width=0.24\linewidth,crop=pdfcrop]{./images/SAFE/SAFE_BS_15x10_HTI_45/P_a_3_3kHz}		
%\end{minipage}
%\begin{minipage}{0.24\linewidth}
%	\psfragfig[width=0.24\linewidth,crop=pdfcrop]{./images/SAFE/SAFE_BS_15x10_HTI_45/P_a_5_5kHz}		
%\end{minipage}
%\begin{minipage}{0.24\linewidth}
%	\psfragfig[width=0.24\linewidth,crop=pdfcrop]{./images/SAFE/SAFE_BS_15x10_HTI_45/P_a_8_6kHz}		
%\end{minipage}
%\vspace{\baselineskip}
%\renewcommand{\arraystretch}{1.0}
%\footnotesize
%\begin{tabular*}{\textwidth}{@{\extracolsep{\fill} }clclcl}
%\tikz \draw (0,0) -- (1cm,0); & ортогональный & \tikz \draw[dashed] (0,0) -- (1cm,0); & ортогональный & \tikz \draw[dashdotted] (0,0) -- (1cm,0); & неортогональный    \\
% & без фильтрации & & с фильтрацией & & с фильтрацией \\
%\end{tabular*}
%\renewcommand{\arraystretch}{1.0}
%\normalsize
%\caption{Результаты расчетов собственных векторов для Модели \ref{mnum: 2}}
%\label{fig:bs15_10_HTI45}
%\end{figure}

%\begin{figure}[h]
%\centering
%\begin{minipage}{0.24\linewidth}
%	\psfragfig[width=0.24\linewidth,crop=pdfcrop]{./images/SAFE/SAFE_CS_15x10_HTI_45/P_s_2kHz}		
%\end{minipage}
%\begin{minipage}{0.24\linewidth}
%	\psfragfig[width=0.24\linewidth,crop=pdfcrop]{./images/SAFE/SAFE_CS_15x10_HTI_45/P_s_3_3kHz}		
%\end{minipage}
%\begin{minipage}{0.24\linewidth}
%	\psfragfig[width=0.24\linewidth,crop=pdfcrop]{./images/SAFE/SAFE_CS_15x10_HTI_45/P_s_5_5kHz}		
%\end{minipage}
%\begin{minipage}{0.24\linewidth}
%	\psfragfig[width=0.24\linewidth,crop=pdfcrop]{./images/SAFE/SAFE_CS_15x10_HTI_45/P_s_8_6kHz}		
%\end{minipage}
%\begin{minipage}{0.24\linewidth}
%	\psfragfig[width=0.24\linewidth,crop=pdfcrop]{./images/SAFE/SAFE_CS_15x10_HTI_45/P_a_2kHz}		
%\end{minipage}
%\begin{minipage}{0.24\linewidth}
%	\psfragfig[width=0.24\linewidth,crop=pdfcrop]{./images/SAFE/SAFE_CS_15x10_HTI_45/P_a_3_3kHz}		
%\end{minipage}
%\begin{minipage}{0.24\linewidth}
%	\psfragfig[width=0.24\linewidth,crop=pdfcrop]{./images/SAFE/SAFE_CS_15x10_HTI_45/P_a_5_5kHz}		
%\end{minipage}
%\begin{minipage}{0.24\linewidth}
%	\psfragfig[width=0.24\linewidth,crop=pdfcrop]{./images/SAFE/SAFE_CS_15x10_HTI_45/P_a_8_6kHz}		
%\end{minipage}
%\vspace{\baselineskip}
%\renewcommand{\arraystretch}{1.0}
%\footnotesize
%\begin{tabular*}{\textwidth}{@{\extracolsep{\fill} }clclcl}
%\tikz \draw (0,0) -- (1cm,0); & ортогональный & \tikz \draw[dashed] (0,0) -- (1cm,0); & ортогональный & \tikz \draw[dashdotted] (0,0) -- (1cm,0); & неортогональный    \\
% & без фильтрации & & с фильтрацией & & с фильтрацией \\
%\end{tabular*}
%\renewcommand{\arraystretch}{1.0}
%\normalsize
%\caption{Результаты расчетов собственных векторов для Модели \ref{mnum: 4}}
%\label{fig:cs15_10_HTI45}
%\end{figure}

%Из представленных данных хорошо заметно, что неортогональная версия Alford rotation дает наиболее близкие оценки направлений поляризации дипольных волн на низких частотах. Этому вероятно способствует малое количество энергии в волнах на частотах ниже 1.5 кГц, а также сравнительно малый вклад узкой переходной области средних частот (3-4 кГц). Эти направления не совпадают с заданной ориентацией оси симметрии трансверсально-изотропной породы, а более близкие значения классического ортогонального алгоритма можно объяснить тем, что полученные ортогональные значения ориентаций ближе всего находятся реальным поляризациям мод на этих частотах. В пользу последнего утверждения также говорит и тот факт, что энергия недиагональных компонент в ортогональном случае приблизительно на 10\% выше. 
%
%В целом же можно твердо сказать, что направления поляризации в рассмотренных задачах полностью определяются ориентацией мод на высоких частотах. Интересно, что ориентация мод на высоких частотах почти ортогональна, но не совпадает геометрически с ориентацией полуосей эллипса поперечного сечения скважины, как в случае с изотропной породой. При увеличении степени эллиптичности ствола это различие сокращается. 

\clearpage
%\bibliography{./library/library}
%\include{include/var_bibliography_link}
\bibliography{c:/Users/German/Documents/TeX_Library/library}
%\bibliography{d:/Documents/Workfiles/Literature/TeX_Library/library}

\bibliographystyle{unsrt}
%\bibliographystyle{plainnat_no_url}

%\appendix
%\section{Фильтрация данных}
%
%\begin{minipage}[c]{0.47\linewidth}	
%\begin{center}
%		Ортогональный Alford rotation \\
%		\psfragfig[width=0.40\linewidth,crop=pdfcrop]{./images/nonorth_alford/el20x10_TTI60_rot4c_scheme}			
%	  		\label{fig:rot4_scheme}
%\end{center}	  		
%\end{minipage} \hfill
%\begin{minipage}[c]{0.47\linewidth}
%\begin{center}
%		Неортогональный Alford rotation\\
%			\psfragfig[width=0.40\linewidth,crop=pdfcrop]{./images/nonorth_alford/el20x10_TTI60_gs_rot4c_scheme}
%			\label{fig:rot4_gs_scheme}
%\end{center}
%\end{minipage} 	\\


\end{document} 